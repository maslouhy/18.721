\documentclass [letterpaper,11pt,twoside]{article}
%\usepackage{pst-pdf,pst-text,pstricks-add}
\providecommand{\ifincludeall}{\iffalse}
\usepackage{fancyhdr}
\usepackage{lastpage}
\usepackage{enumerate}
\ifincludeall

  \usepackage{wrapfig}

%================================ Unicode ================================
  \usepackage[utf8]{inputenc}
  \DeclareUnicodeCharacter{916}{\ensuremath{\Delta}}
  \DeclareUnicodeCharacter{937}{\ensuremath{\Omega}}
  \DeclareUnicodeCharacter{949}{\ensuremath{\epsilon}}
  \DeclareUnicodeCharacter{956}{\ensuremath{\mu}}
  \DeclareUnicodeCharacter{963}{\ensuremath{\sigma}}
  \DeclareUnicodeCharacter{977}{\ensuremath{\theta}}
  \DeclareUnicodeCharacter{1009}{\ensuremath{\rho}}
  \DeclareUnicodeCharacter{03B4}{\ensuremath{\delta}}
  \DeclareUnicodeCharacter{221A}{\sqrt}
  \DeclareUnicodeCharacter{2124}{\ensuremath{\mathbb Z}}
%============================== End Unicode ==============================
\fi


\usepackage{amsmath}
\usepackage{amssymb}
%================================= AMSTHM =================================
\usepackage{amsthm}

\newtheorem{thm}{Theorem}[section]
%\newtheorem{theorem}{Theorem}
\newtheorem{conjecture}{Conjecture}[section]
\newtheorem{lem}{Lemma}[section]
%\newtheorem{lemma}[theorem]{Lemma}
\newtheorem{cor}[thm]{Corollary}
%\newtheorem{corollary}[theorem]{Corollary}
\newtheorem{prop}[thm]{Proposition}
%\newtheorem{proposition}[theorem]{Proposition}
%\newtheorem{definition}[theorem]{Definition}
%\newtheorem{example}[theorem]{Example}
\newtheorem{exercise}[thm]{Exercise}
%\newtheorem{exercise}[theorem]{Exercise}
\newtheorem{claim}[thm]{Claim}
\newtheorem{law}{Law}[section]
\newtheorem*{thm*}{Theorem}
\newtheorem*{lem*}{Lemma}
\newtheorem*{conjecture*}{Conjecture}
\newtheorem*{cor*}{Corollary}
\newtheorem*{prop*}{Proposition}
\newtheorem*{exercise*}{Exercise}
\newtheorem*{law*}{Law}
\newtheorem*{claim*}{Claim}



\theoremstyle{definition} \newtheorem{defn}{Definition}[section]
\theoremstyle{definition} \newtheorem*{defn*}{Definition}
\newtheorem{example}[thm]{Example}
\newtheorem*{example*}{Example}
\newtheorem{eg}[thm]{Example}
\newtheorem*{eg*}{Example}

\newtheorem{fact}{Fact}[section]
\newtheorem*{fact*}{Fact}

\newcommand{\thmref}[1]{Theorem~\ref{#1}}
%=============================== End AMSTHM ===============================
\usepackage{esint}
\ifincludeall
  \usepackage[table]{xcolor}
  \usepackage{xifthen}
\fi
%\if dcpic
%\usepackage{dcpic}
%\else

%\ifincludeall
  \usepackage{ifpdf}
%\else
  %\newif\ifpdf
  %\pdftrue
%\fi

\ifincludeall
  \ifpdf
    \usepackage[pdftex]{graphicx}
    \usepackage{pdfpages}
    \usepackage[plainpages=false,pdfpagelabels,unicode]{hyperref}
  \else
    \usepackage[dvips]{graphicx}
    \usepackage{pstricks,pstricks-add,pst-math,pst-xkey}
  \fi
\else
  \usepackage[pdftex]{graphicx}
  \usepackage[pdfpagelabels,unicode]{hyperref}
\fi

%\ifincludeall
  \usepackage{mathtools}
%\fi

%\global\def\isxy{}
%\global\let\isxy\relax
\ifincludeall
  \providecommand{\isxy}{\let\isxy\relax}
  \expandafter\ifx\isxy\relax
    %\usepackage{mathpazo} or \usepackage{mathptmx} 
    \usepackage{flexisym} % flexisym package is required by breqn
                          % can load as \usepackage[mathpazo]{flexisym}
    \usepackage{breqn}
  \else
    \usepackage[all]{xy}
  \fi
\fi



%\usepackage[exponent-product=\cdot,per-mode=fraction,quotient-mode=fraction,fraction-function=\sfrac]{siunitx} %alsoload={named,prefixed,abbr,hep},
%\newunit{\statvolt}{statV}
%\newunit{\erg}{erg}
%\newunit{\esu}{esu}

\providecommand{\abs}[1]{\left\lvert#1\right\rvert}%\DeclarePairedDelimiter\abs{\lvert}{\rvert} %\providecommand{\abs}[1]{\lvert#1\rvert}
\ifincludeall
  \DeclarePairedDelimiter\norm{\lVert}{\rVert}
  \DeclarePairedDelimiter\floor{\lfloor}{\rfloor}
\else
  \providecommand{\floor}[1]{\left\lfloor #1\right\rfloor}
  \providecommand{\norm}[1]{\lVert#1\rVert}
\fi
\newcommand{\gcdf}[2]{\left( #1 , #2 \right)}
\DeclareMathOperator{\lcm}{lcm}
\DeclareMathOperator{\im}{im}
\DeclareMathOperator{\rank}{rank}
\DeclareMathOperator{\spans}{span}
\DeclareMathOperator{\divergence}{div}
\DeclareMathOperator{\tr}{tr}
\DeclareMathOperator{\grad}{grad}
\DeclareMathOperator{\spec}{Spec}
\DeclareMathOperator{\pspec}{PSpec}
\DeclareMathOperator{\rad}{rad}
\DeclareMathOperator{\trdeg}{tr\,deg}
\DeclareMathOperator{\gk}{gk}
\DeclareMathOperator{\fract}{fract}
\newcommand{\lcmf}[2]{\lcm\left( #1 , #2 \right)}
\def\dbar{{\mathchar'26\mkern-12mu d}}

\def\sfrac#1/#2{\leavevmode\kern.1em\raise.5ex\hbox{\the\scriptfont0 #1}\kern-.1em/\kern-.15em\lower.25ex\hbox{\the\scriptfont0 #2}}

\renewcommand{\d}{\,d}

\newcommand{\defeq}{\coloneqq}%\stackrel{\mathrm{df}}{=}}%{\ensuremath{:=}}%

\ifincludeall
  \newcommand{\complementset}[1][\ \ \rule{0pt}{1ex}]{\overline{#1}}
  \newcommand{\boldcomplementset}[1][\ \ \rule{0pt}{1ex}]{\text{\makebox[0pt][l]{$#1$}}\rule[\heightof{#1}+3pt]{\widthof{#1}}{0.11ex}}
  \let\oldboldsymbol=\boldsymbol
  \renewcommand{\boldsymbol}[1]{\let\oldcomplementset=\complementset%
  %\oldboldsymbol{#1}\  	%
  \let\complementset\boldcomplementset%
  \oldboldsymbol{#1}%
  \let\complementset=\oldcomplementset}
\fi
% \gdef\phantomhdepth{\relax}
% \gdef\phantomhdepth{1}
% \newcommand{\phantomheight}[1]{         %
% \ifx\phantomhdepth\relax          %
%   \let\phantomhdepth{1}          %
%   \vphantom{#1}          %
%   \let\phantomhdepth\relax          %
% \fi          %
% }
\newcommand{\tuple}[1]{\breakingtuple{#1}}
\newcommand{\nbtuple}[1]{\left(#1\right)}
\newcommand{\breakingtuple}[1]{\lrbreak{(}{#1}{)}}         %$
         %\let\oldcomma=,
\begingroup
  \lccode`~=`,
  \lowercase{\endgroup
    \let\oldcomma=~
    \def\comma{\oldcomma}
    \def~{\comma}         %
  }         %
         %\def\aaa{\comma}
         %\mathcode`,="613B

\newcommand{\allowbreaks}[1]{\begingroup \mathcode`,="8000 \def\comma{\oldcomma\allowbreak}#1\def\comma{\oldcomma} \mathcode`,="613B \endgroup }         %\replace{#1}{,}{,\allowbreak}}
%\let\phantomheight\vphantom
\newcommand{\lbreakh}[4][\!\!]{\left#2\vphantom{#3#4}\right.#1\allowbreaks{#3}}
\newcommand{\rbreakh}[4][\!\!]{#2#1\left.\vphantom{#2#4}\right#3}
\newcommand{\lrbreakh}[5][\!\!]{\left#2\vphantom{#3#5}\right.#1\allowbreaks{#3}#1\left.\vphantom{#3#5}\right#4}
\newcommand{\lbreak}[3][\!\!]{\lbreakh[#1]{#2}{#3}{}}
\newcommand{\rbreak}[3][\!\!]{\rbreakh[#1]{#2}{#3}{}}
\newcommand{\lrbreak}[4][\!\!]{\lrbreakh[#1]{#2}{#3}{#4}{}}
\newcommand{\olrbreak}[5][\!\!]{\overlineb{\lrbreakh[#1]{#2}{#3}{.}{#4}}{\lrbreakh[#1]{.}{#4}{#5}{#3}}}
\newcommand{\olrbreakc}[6][\!\!]{\overlineb{#2\lbreakh[#1]{#3}{#4}{#5}}{\rbreakh[#1]{#5}{#6}{#4}}}%\lrbreak[#1]{.}{#5}{#5}}}

%\DeclarePairedDelimiter\simpleset{\lbrace}{\rbrace}
\newcommand{\simpleset}[1]{\left\lbrace#1\right\rbrace}
\newcommand{\simplesetb}[1]{\lrbreak{\lbrace}{#1}{\rbrace}}

\newcommand{\mathsetb}[2][\relax]{%
\ifx#1\relax
  \simplesetb{#2}
\else
  \simplesetb{\suchthatb[#1]{#2}}
\fi}

\newcommand{\mathset}[2][\relax]{%
\ifx#1\relax
  \simpleset{#2}
\else
  \simpleset{\suchthat[#1]{#2}}
\fi}
\newcommand{\omathset}[3][\relax]{%
\ifx#1\relax
  \overlineb{\lrbreak{\lbrace}{#2}{.}}{\lrbreak{.}{#3}{\rbrace}}
\else
  \overlineb{\lrbreak{\lbrace}{\suchthat[#1]{#2}}{.}}{\lrbreak{.}{#3}{\rbrace}}
\fi}

\newcommand{\omathsetc}[4][\relax]{%
\ifx#1\relax
  \overlineb{\lrbreak{\lbrace}{\suchthat[#2]{#3}}{.}}{\lrbreak{.}{#4}{\rbrace}}
\else
  \overlineb{#1\lrbreak{\lbrace}{\suchthat[#2]{#3}}{.}}{\lrbreak{.}{#4}{\rbrace}}
\fi}
  
\newcommand{\overlinet}[2]{\overlineb{#1}{#2}}
\newcommand{\overlineb}[2]{\overline{#1\vphantom{#2}}\allowbreak\overline{#2\vphantom{#1}}}
%\newcommand{\overlinec}[3]{\overline{#1\text{\makebox[0pt]{$\phantom{#2#3}$}}}\allowbreak\overline{#2\vphantom{#1#3}}\allowbreak\overline{#3\vphantom{#1#2}}}

\newcommand{\suchthat}[2][]{#1\left\vert\vphantom{#1#2}\right.\!#2}
\newcommand{\suchthatb}[2][]{#1 \lbreakh[\!]{\vert}{#2}{#1}}
\providecommand{\scfont}{}

%texbook
%\newif\ifv@ \newif\ifh@
%\def\vphantom{\v@true\h@false\ph@nt}
%\def\hphantom{\v@false\h@true\ph@nt}
%\def\phantom{\v@true\h@true\ph@nt}
%\def\ph@nt{\ifmmode\def\next{\mathpalette\mathph@nt}%
%\else\let\next=\makeph@nt\fi \next}
%\def\makeph@nt#1{\setbox0=\hbox{#1}\finph@nt}
%\def\mathph@nt#1#2{\setbox0=\hbox{$\m@th#1{#2}$}\finph@nt}
%\def\finph@nt{\setbox2=\null \ifv@ \ht2=\ht0 \dp2=\dp0 \fi
%\ifh@ \wd2=\wd0 \fi \box2 }


%\def\m@th{\mathsurround=0pt }
\def\uncurry#1#2{#1#2} % \uncurry\macro1{{arg1}{arg2}...} -> \macro1{arg1}{arg2}...
\def\curryone#1#2#3{{\def\ndef{\noexpand\def}\def\first{\noexpand#1}\expandafter}\ifx\ndef\first #1{#2{#3}}\else #1{{#2}{#3}}\fi}
\def\currytwo#1#2#3#4{#1{{#2}{#3}{#4}}}
\def\currythree#1#2#3#4#5{#1{{#2}{#3}{#4}{#5}}}
\def\curryfour#1#2#3#4#5#6{#1{{#2}{#3}{#4}{#5}{#6}}}
\def\curryfive#1#2#3#4#5#6#7{#1{{#2}{#3}{#4}{#5}{#6}{#7}}}
\def\currysix#1#2#3#4#5#6#7#8{#1{{#2}{#3}{#4}{#5}{#6}{#7}{#8}}}
\def\curryseven#1#2#3#4#5#6#7#8#9{#1{{#2}{#3}{#4}{#5}{#6}{#7}{#8}{#9}}}


\def\uncurrytwo#1#2#3{#1#2#3}
\def\uncurriedmathpalette#1{\def\uncurried{\uncurrytwo#1}\mathpalette\uncurried}
\def\mathpalettetwo#1#2#3{\uncurriedmathpalette{#1}{{#2}{#3}}}
\def\mathpalettethree#1#2#3#4{\uncurriedmathpalette{#1}{{#2}{#3}{#4}}}


%\def\aswidthof{\v@false\h@true\@sof}
%\def\asheightof{\v@true\h@false\@sof}
%\def\assizeof{\v@true\h@true\@sof}

%\newcommand{\@sof}[3][l]{\ifmmode \def\next{\mathpalettethree\math@sof}%
%\else\let\next=\make@sof\fi \next{#1}{#2}{#3}}
%
%\def\make@sof#1#2#3{\setbox0=\hbox{#2}\setbox2=\makebox[0pt][c]{#3}\fin@sof}
%\def\math@sof#1#2#3{\setbox0=\hbox{$\m@th#1{#2}$}\setbox2=\hbox{$\m@th#1{#3}$}\fin@sof}
%\def\fin@sof{\ifv@ \ht2=\ht0 \dp2=\dp0 \fi
%\ifh@ \wd2=\wd0 \fi \box2 }
\makeatletter
\newcommand{\aswidthof}[3][c]{\ifmmode \def\next{\mathpalettethree\m@th@swidthof}%
\else\let\next=\m@ke@swidthof\fi \next{#1}{#2}{#3}}

\newdimen\@widthof
\newcommand{\m@th@swidthof}[4]{\text{\settowidth\@widthof{$\m@th#1#3$}\makebox[\@widthof][#2]{$\m@th#1#4$}}}
\newcommand{\m@ke@swidthof}[3]{\settowidth\@widthof{#2}\makebox[\@widthof][#1]{#3}}


%
%
%
%
%
%\makeatletter
%
%\def\aswidthof{\v@false\h@true\@sof}
%\def\asheightof{\v@true\h@false\@sof}
%\def\assizeof{\v@true\h@true\@sof}
%
%\def\@sof{\ifmmode\def\next##1##2{\mathpalette{\math@sof##1##2}}%
%\else\let\next=\make@sof\fi \next}
%
%\def\make@sof#1#2{\setbox0=\hbox{#1}\setbox2=\hbox{#2}\fin@sof}
%\def\math@sof#1#2#3{\setbox0=\hbox{$\m@th#3{#1}$}\setbox2=\hbox{$\m@th#3{#2}$}\fin@sof}
%\def\fin@sof{\ifv@ \ht2=\ht0 \dp2=\dp0 \fi
%\ifh@ \wd2=\wd0 \fi \box2 }

%\text{\newdimen\tempwidth \settowidth\tempwidth{$#3$}\makebox[\tempwidth][#1]{$#2$}}}% \text{\raisebox{0ex}[-\height][-\height]{$\phantom{#2}$}}}
%\newcommand{\asheightof}[3]{#1\text{\makebox[0pt]{$\phantom{#2}$}}\right.#2\left.\text{\makebox[0pt]{$\phantom{#2}$}}#3}

% 276887 sp is twice the width difference between $\left(\right)$ and $\left(\right.\left.\right)$
% \hspace{-138444 sp} \hspace{-138443 sp}\
\makeatletter
\def\breakingleft#1{{\def\templeft##1\breakingright##2{\left#1\vphantom{##1}\right.\n@space##1\n@space\left.\vphantom{##1}\right##2}\expandafter}\templeft}

\newcommand{\fullexpand}[2][]{{#1\edef\temp{#2}\expandafter}\temp}
\def\settoksexpanded#1=#2{{\edef\temp{{#2}}\expandafter}\expandafter#1\expandafter=\temp}

\newcommand{\selectfontsize}[1]{\fontsize{#1}{#1}\selectfont}

\newcommand{\newlinep}{$\left.\right.$\par\noindent}

\makeatletter
\newcommand{\interitemtext}[1]{%
\begin{list}{}
{\itemindent=0mm\labelsep=0mm
\labelwidth=0mm\leftmargin=0mm
\addtolength{\leftmargin}{-\@totalleftmargin}}
\item #1
\end{list}}
\makeatother


\newcommand{\ncr}[2]{\binom{#1}{#2}}
\newcommand{\definition}[1]{\begin{defn}#1\end{defn}}%{\par \noindent {\bf Definition:} #1}
\newcommand{\factc}[1]{\par \noindent \underline{\bf Fact:} #1}
%\providecommand{\lrang}[1]{\left\langle#1\right\rangle}%\DeclarePairedDelimiter\lrang{\langle}{\rangle}
\providecommand{\WLOG}{Without loss of generality}
\providecommand{\TFAE}{The following are equivalent}%{These facts are excellent}%(Ari)
\def\<#1>{\left\langle#1\right\rangle}
\def\[#1]{\left[#1\right]}
%\def\{#1|#2}{\mathset[#1]{#2}}
\providecommand{\End}[1]{\mathop{End}\left(#1\right)}
\providecommand{\dom}[1][\relax]{\mathop{dom}%
\ifx#1\relax%
\else%
  \left(#1\right)%
\fi}

\providecommand{\ses}{short exact sequence}
\providecommand{\st}{such that}


\newenvironment{digression}{\noindent\begin{math}%
\left(\ \begin{minipage}{0.95\textwidth}% FIX: Make this more robust
  }{\end{minipage}\ \right)\end{math}}

%\newenvironment{correction}{\textcolor{red}}{}
%======================================== Physics ===================================================
\providecommand{\uvec}[1]{{\widehat{\bf{#1}}}}
%\MakeAtOther
%\providecommand{\declareunits}[1][km,m,cm,mm,s,L,mL,kg,g]{%
%\@for\@unit:=#1\do{%
%  \declareunit{\@unit}%
%}}
%\providecommand{\declareunit}[2][\relax]{\ifx#1\relax%
%  \def\csname #2\endcsname{\text{#2}}%
%\else
%  \def\csname #1\endcsname{\text{#2}}%
%\fi}
%====================================================================================================

%================================= Proof Cases ===============================
\newcounter{ProofCasesLvlCtr}
\newcounter{ProofCasesMaxCtr}
\newcounter{ProofCasesCurCtr}
\newenvironment{proof-cases}[1][1]{%
\setcounter{ProofCasesCurCtr}{\value{ProofCasesLvlCtr}}%
\ifthenelse{\arabic{ProofCasesMaxCtr} = \arabic{ProofCasesLvlCtr}}{%
  \stepcounter{ProofCasesLvlCtr}%
  \stepcounter{ProofCasesMaxCtr}%
  \expandafter\newcounter{ProofCasesCtr\arabic{ProofCasesLvlCtr}}%
  \ifthenelse{\arabic{ProofCasesLvlCtr} = 1}{%
    \expandafter\edef\csname labelProofCases\arabic{ProofCasesLvlCtr}\endcsname{Case~\noexpand\arabic{ProofCasesCtr\arabic{ProofCasesLvlCtr}}}%
  }{%
    \expandafter\edef\csname labelProofCases\arabic{ProofCasesLvlCtr}\endcsname{\csname labelProofCases\arabic{ProofCasesCurCtr}\endcsname.\noexpand\arabic{ProofCasesCtr\arabic{ProofCasesLvlCtr}}}%
  }%
}{%
  \stepcounter{ProofCasesLvlCtr}%
  \ifthenelse{\arabic{ProofCasesLvlCtr} = 1}{%
    \expandafter\edef\csname labelProofCases\arabic{ProofCasesLvlCtr}\endcsname{Case~\noexpand\arabic{ProofCasesCtr\arabic{ProofCasesLvlCtr}}}%
  }{%
    \expandafter\edef\csname labelProofCases\arabic{ProofCasesLvlCtr}\endcsname{\csname labelProofCases\arabic{ProofCasesCurCtr}\endcsname.\noexpand\arabic{ProofCasesCtr\arabic{ProofCasesLvlCtr}}}%
  }%
}%
\begin{list}{\csname labelProofCases\arabic{ProofCasesLvlCtr}\endcsname:}{\expandafter\usecounter{ProofCasesCtr\arabic{ProofCasesLvlCtr}}}%
\expandafter\setcounter{ProofCasesCtr\arabic{ProofCasesLvlCtr}}{#1-1}%
}{\end{list}\addtocounter{ProofCasesLvlCtr}{-1}}
%\usepackage[pointlessenum]{paralist}
%\newenvironment{proof-cases}{\begin{enumerate}[{Case} 1:]}{\end{enumerate}}
%=============================== End Proof Cases =============================

\allowdisplaybreaks[1]

\usepackage[margin=1in]{geometry}

\usepackage{cancel}
% email to abhinav@math.mit.edu
\def\d{\, {\rm d}}

%\usepackage{pgf,tikz}
%\usetikzlibrary{arrows}
\usepackage{wasysym}
\usepackage{pdfcomment}

\usepackage{datetime}
\usepackage{verbatim}
\usepackage[all]{xy}
\def\classnumber{18.721}
\def\classname{Algebraic Geometry}
\edef\cheadcontent{\classnumber\space Notes}

\pagestyle{fancy}
\headheight 13.7pt
\fancyhead{}
\fancyhead{Jason Gross}%\today}
\fancyfoot{}
\lhead{Jason Gross}
\rhead{\TeX ed on \today}
\chead{\cheadcontent}
\cfoot{\thepage\space of \pageref{LastPage}}

\newcommand{\flag}[2][]{#2\footnote{#1}}

\begin {document}
\setcounter {section}{10}\section {Monday, February 28, 2011}
  \begin{itemize}
    \item[] $B$ finite type, $G$ operator(?)
    \item[] $A = B^G$
    \item[] Showed $A$ finite type
    \item[] $Y = \spec B$
    \item[] $X = \spec A$
    \item[] $X \leftrightarrow \{G\text{ orbits in }Y\}$
    \item[] $G \times B \to B$
    \item[] $\sigma, b \leadsto \sigma(b)$
    \item[] (left) $\sigma \tau(b)$: first $\tau$, then $\sigma$
    \item[] Then $G$ operates on the \emph{right} on $Y$.
    \item[] $q\in Y$, $\sigma$ sends $q\to q^\sigma$
    \item[] $q^{\sigma\tau}$: first $\sigma$, then $\tau$
  \end{itemize}
  View
  \begin{align*}
    y & \leftrightarrow \{\text{homomorphisms }B \to \mathbb C\} \\
    q & \leftrightarrow \pi_q: B \to \mathbb C \\
    & \leftrightarrow \{\text{max ideals of }B\} \\
    q & \leftrightarrow m_q
  \end{align*}
  Operation on $Y$:
  $$
  \xymatrix{
    B \ar[r]^\sigma \ar[dr]_{\pi \circ \sigma} & B \ar[d]^\pi \\
                                               & \mathbb C
  }
  $$

  $\pi_q \circ \sigma(b) = \pi_q(\sigma b)$

  Define $q^\sigma = $ that point such that $\pi_{q^\sigma} = \pi_q \circ \sigma$.

  \noindent \emph{Operation on max ideals}: \\
  $\mathfrak M_{q^\sigma} = \sigma^{-1}\mathfrak M_q$ \\
  $Y \to X$? \\
  For any $p\in X$,
  $$
  \xymatrix{
    B \ar[r]^{\pi \circ p} & \mathbb C \\ % fix
    A \ar@{-->}[ur]_{\pi_p}  \ar@{^{(}->}[u]
  }
  $$
  $Y \to X$ sends $q  \leadsto r$ \\
  $$
  \xymatrix{
    B \ar[r]_{\sigma} \ar@/^/[rr]^{\pi} & B \ar[r]_{\pi_q} & \mathbb C \ar@{=}[d] \\
    A \ar@{^{(}->}[u] \ar[r]^{\text{id}} \ar@/_/[rr]_{\pi_p} & A  \ar@{^{(}->}[u] \ar[r]^{\pi_p} & \mathbb C
  }
  $$
  Therefore, $G$-orbits in $Y$ map to points of $X$.

  We want to show that different orbits $\{q_1, \ldots, q_r\} \ne \{q_1', \ldots, q_s'\}$ in $Y$ map to different points $p$, $p'$ in $X$.

  \begin{proof}
    \emph{Plan}: Find an element $a\in A$ such that $a = 0$ on orbit $\{q_i\}$, $\pi_q(a) = 0$, $a \ne 0$ on orbit $\{q'_j\}$.  Then $\pi_{q'_j}(a) \ne 0$.  This would give us that $a\in \mathfrak M_{q_i}$ (same as $\in \mathfrak M_p$) and $a\notin \mathfrak M_{q'_j}$ (same as $\notin \mathfrak M_{p'}$).

    In $B$, choose $b\in \mathfrak M_{q_1}$ (then ) such that $b\notin \mathfrak M_{q'_j}$ for all $j = 1, \ldots, s$.  (Note: $b(q) \defeq \pi_q(b)$.)

    \begin{digression}
      Diversion: Suppose $B = \mathbb C[x_1, \ldots, x_n] / (f_1, \ldots, f_k)$.  $b$ represented by the polynomial $p(x_1, \ldots, x_n)$.  $\spec B \approx V(f_1, \ldots, f_k)$ in $\mathbb C^n$.

      $\spec B = (\text{max ideals}) = (\text{homomorphisms }B \to \mathbb C) = (V(I)\text{ if }B = \mathbb C[x] / I)$)
    \end{digression}

    We can do this. (Think about choosing (hyper-?)planes that do not pass through finite sets of points.)

    Let $a = \displaystyle \prod_{\sigma\in G} \sigma(b)$.  $a$ is invariant.  $a = 0$ on $q_1$ because $b$ divides $a$ in $B$ (since some $\sigma \in G$ is the identity).  Therefore $a = 0$ on the orbit $(q_1)$.  $a \ne 0$ on $q'_1$.

    $\sigma(b)$ evaluated at $q'_1$ is $\pi_{q'_1}(\sigma b) = \pi_{{q'}_1^\sigma}(b) = b$ evaluated at ${q'}_1^\sigma$.

    Therefore, $a \ne 0$ on the orbit.
  \end{proof}

  \subsection{Localization}
    Note: we always assume that the rings are domains and assume (whenever possible) that they're finite type algebras.

    \begin{defn*}
      A \emph{multiplicative system} $S$ in a domain $A$ is a subset of $A$ satisfying
      \begin{itemize}
        \item $1 \in S$
        \item $0 \notin S$
        \item if $s, t\in S$, then $st \in S$
      \end{itemize}
    \end{defn*}
    \begin{defn*}
      The elements of $S$ serve as denominators in the \emph{ring of fractions}
      $$A_S \defeq \left.\mathset[ \frac{a}{s} ]{s\in S, a\in A}\right/\sim$$
      where $\frac{a}{s} \sim \frac{b}{t}$ if $at = bs$
    \end{defn*}
    \begin{align*}
      A & \hookrightarrow A_S \\
      a & \leadsto a / 1
    \end{align*}
    \begin{example*}
      $S = \{1, s, s^2, \ldots\}$, $s \ne 0$.  $A_S = A[s^{-1}] = A[y] / (sy - 1)$
    \end{example*}
    \begin{example*}
      $S = A - \{0\}$, $A_S = $ fraction field
    \end{example*}
    \begin{example*}
      $P$ a prime ideal of $A$, $S = A - P = \mathset[s]{s\notin P}$.  Then $s\notin P$, $t\notin P$ $\implies$ $st\notin P$.  Then $A_S$ is the \emph{localization of $A$ at $P$}.  This is (perversely) denoted $A_P$.
    \end{example*}

    If $A \subset B$ a subring, then we can relate ideals of $A$ and $B$: \\
    \emph{Extended ideal}: $I^e$
    \begin{itemize}
      \item[] $I$ ideal of $A$
      \item[] $IB = $ ideal of $B$ generated by $\{I\}$
      \item[] The elements are $$\sum_{\text{finite}} x_i b_i\qquad,\ x_i\in I,\ b_i\in B$$
    \end{itemize}
    \emph{Contracted ideal}: $J^c$: For $J$ an ideal of $B$, $(J \cap A) = $ ideal of $A$

    \begin{itemize}
      \item[] $(I^e)^c \supset I$
      \item[] $(J^c)^e \subset J$
    \end{itemize}

    For $A \subset B = A_S$:
    \begin{itemize}
      \item[] $I^e = IA_s = \mathset[x / s]{x\in I,\ s\in S} / \sim$
    \end{itemize}

    $J^c = J \cap A$.  If $y / s\in J$, then $y\in J \cap A = J^c$.  Therefore, $y / s\in (J^c)^e$.  Thus, $J \subset (J^c)^e$, so $J = (J^c)^e$.

    \begin{cor*}
      If $A$ noetherian, then $A_S$ noetherian
    \end{cor*}
    \begin{proof}
      Take an increasing sequence $J_1 \subset J_2 \subset \cdots$ of ideals in $A_S$.  Let $I_\nu = J_\nu \cap A$.  Then $I_1 \subset I_2 \subset \cdots$.  Since $A$ is noetherian, this is eventually constant.  Therefore $I_\nu^e = (J_\nu^c)^e$ eventually constant.  Thus $A_S$ is noetherian.
    \end{proof}
\end {document}

