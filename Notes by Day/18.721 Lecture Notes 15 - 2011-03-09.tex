\documentclass [letterpaper,11pt,twoside]{article}
%\usepackage{pst-pdf,pst-text,pstricks-add}
\providecommand{\ifincludeall}{\iffalse}
\usepackage{fancyhdr}
\usepackage{lastpage}
\usepackage{enumerate}
\ifincludeall

  \usepackage{wrapfig}

%================================ Unicode ================================
  \usepackage[utf8]{inputenc}
  \DeclareUnicodeCharacter{916}{\ensuremath{\Delta}}
  \DeclareUnicodeCharacter{937}{\ensuremath{\Omega}}
  \DeclareUnicodeCharacter{949}{\ensuremath{\epsilon}}
  \DeclareUnicodeCharacter{956}{\ensuremath{\mu}}
  \DeclareUnicodeCharacter{963}{\ensuremath{\sigma}}
  \DeclareUnicodeCharacter{977}{\ensuremath{\theta}}
  \DeclareUnicodeCharacter{1009}{\ensuremath{\rho}}
  \DeclareUnicodeCharacter{03B4}{\ensuremath{\delta}}
  \DeclareUnicodeCharacter{221A}{\sqrt}
  \DeclareUnicodeCharacter{2124}{\ensuremath{\mathbb Z}}
%============================== End Unicode ==============================
\fi


\usepackage{amsmath}
\usepackage{amssymb}
%================================= AMSTHM =================================
\usepackage{amsthm}

\newtheorem{thm}{Theorem}[section]
%\newtheorem{theorem}{Theorem}
\newtheorem{conjecture}{Conjecture}[section]
\newtheorem{lem}{Lemma}[section]
%\newtheorem{lemma}[theorem]{Lemma}
\newtheorem{cor}[thm]{Corollary}
%\newtheorem{corollary}[theorem]{Corollary}
\newtheorem{prop}[thm]{Proposition}
%\newtheorem{proposition}[theorem]{Proposition}
%\newtheorem{definition}[theorem]{Definition}
%\newtheorem{example}[theorem]{Example}
\newtheorem{exercise}[thm]{Exercise}
%\newtheorem{exercise}[theorem]{Exercise}
\newtheorem{claim}[thm]{Claim}
\newtheorem{law}{Law}[section]
\newtheorem*{thm*}{Theorem}
\newtheorem*{lem*}{Lemma}
\newtheorem*{conjecture*}{Conjecture}
\newtheorem*{cor*}{Corollary}
\newtheorem*{prop*}{Proposition}
\newtheorem*{exercise*}{Exercise}
\newtheorem*{law*}{Law}
\newtheorem*{claim*}{Claim}



\theoremstyle{definition} \newtheorem{defn}{Definition}[section]
\theoremstyle{definition} \newtheorem*{defn*}{Definition}
\newtheorem{example}[thm]{Example}
\newtheorem*{example*}{Example}
\newtheorem{eg}[thm]{Example}
\newtheorem*{eg*}{Example}

\newtheorem{fact}{Fact}[section]
\newtheorem*{fact*}{Fact}

\newcommand{\thmref}[1]{Theorem~\ref{#1}}
%=============================== End AMSTHM ===============================
\usepackage{esint}
\ifincludeall
  \usepackage[table]{xcolor}
  \usepackage{xifthen}
\fi
%\if dcpic
%\usepackage{dcpic}
%\else

%\ifincludeall
  \usepackage{ifpdf}
%\else
  %\newif\ifpdf
  %\pdftrue
%\fi

\ifincludeall
  \ifpdf
    \usepackage[pdftex]{graphicx}
    \usepackage{pdfpages}
    \usepackage[plainpages=false,pdfpagelabels,unicode]{hyperref}
  \else
    \usepackage[dvips]{graphicx}
    \usepackage{pstricks,pstricks-add,pst-math,pst-xkey}
  \fi
\else
  \usepackage[pdftex]{graphicx}
  \usepackage[pdfpagelabels,unicode]{hyperref}
\fi

%\ifincludeall
  \usepackage{mathtools}
%\fi

%\global\def\isxy{}
%\global\let\isxy\relax
\ifincludeall
  \providecommand{\isxy}{\let\isxy\relax}
  \expandafter\ifx\isxy\relax
    %\usepackage{mathpazo} or \usepackage{mathptmx} 
    \usepackage{flexisym} % flexisym package is required by breqn
                          % can load as \usepackage[mathpazo]{flexisym}
    \usepackage{breqn}
  \else
    \usepackage[all]{xy}
  \fi
\fi



%\usepackage[exponent-product=\cdot,per-mode=fraction,quotient-mode=fraction,fraction-function=\sfrac]{siunitx} %alsoload={named,prefixed,abbr,hep},
%\newunit{\statvolt}{statV}
%\newunit{\erg}{erg}
%\newunit{\esu}{esu}

\providecommand{\abs}[1]{\left\lvert#1\right\rvert}%\DeclarePairedDelimiter\abs{\lvert}{\rvert} %\providecommand{\abs}[1]{\lvert#1\rvert}
\ifincludeall
  \DeclarePairedDelimiter\norm{\lVert}{\rVert}
  \DeclarePairedDelimiter\floor{\lfloor}{\rfloor}
\else
  \providecommand{\floor}[1]{\left\lfloor #1\right\rfloor}
  \providecommand{\norm}[1]{\lVert#1\rVert}
\fi
\newcommand{\gcdf}[2]{\left( #1 , #2 \right)}
\DeclareMathOperator{\lcm}{lcm}
\DeclareMathOperator{\im}{im}
\DeclareMathOperator{\rank}{rank}
\DeclareMathOperator{\spans}{span}
\DeclareMathOperator{\divergence}{div}
\DeclareMathOperator{\tr}{tr}
\DeclareMathOperator{\grad}{grad}
\DeclareMathOperator{\spec}{Spec}
\DeclareMathOperator{\pspec}{PSpec}
\DeclareMathOperator{\rad}{rad}
\DeclareMathOperator{\trdeg}{tr\,deg}
\DeclareMathOperator{\gk}{gk}
\DeclareMathOperator{\fract}{fract}
\newcommand{\lcmf}[2]{\lcm\left( #1 , #2 \right)}
\def\dbar{{\mathchar'26\mkern-12mu d}}

\def\sfrac#1/#2{\leavevmode\kern.1em\raise.5ex\hbox{\the\scriptfont0 #1}\kern-.1em/\kern-.15em\lower.25ex\hbox{\the\scriptfont0 #2}}

\renewcommand{\d}{\,d}

\newcommand{\defeq}{\coloneqq}%\stackrel{\mathrm{df}}{=}}%{\ensuremath{:=}}%

\ifincludeall
  \newcommand{\complementset}[1][\ \ \rule{0pt}{1ex}]{\overline{#1}}
  \newcommand{\boldcomplementset}[1][\ \ \rule{0pt}{1ex}]{\text{\makebox[0pt][l]{$#1$}}\rule[\heightof{#1}+3pt]{\widthof{#1}}{0.11ex}}
  \let\oldboldsymbol=\boldsymbol
  \renewcommand{\boldsymbol}[1]{\let\oldcomplementset=\complementset%
  %\oldboldsymbol{#1}\  	%
  \let\complementset\boldcomplementset%
  \oldboldsymbol{#1}%
  \let\complementset=\oldcomplementset}
\fi
% \gdef\phantomhdepth{\relax}
% \gdef\phantomhdepth{1}
% \newcommand{\phantomheight}[1]{         %
% \ifx\phantomhdepth\relax          %
%   \let\phantomhdepth{1}          %
%   \vphantom{#1}          %
%   \let\phantomhdepth\relax          %
% \fi          %
% }
\newcommand{\tuple}[1]{\breakingtuple{#1}}
\newcommand{\nbtuple}[1]{\left(#1\right)}
\newcommand{\breakingtuple}[1]{\lrbreak{(}{#1}{)}}         %$
         %\let\oldcomma=,
\begingroup
  \lccode`~=`,
  \lowercase{\endgroup
    \let\oldcomma=~
    \def\comma{\oldcomma}
    \def~{\comma}         %
  }         %
         %\def\aaa{\comma}
         %\mathcode`,="613B

\newcommand{\allowbreaks}[1]{\begingroup \mathcode`,="8000 \def\comma{\oldcomma\allowbreak}#1\def\comma{\oldcomma} \mathcode`,="613B \endgroup }         %\replace{#1}{,}{,\allowbreak}}
%\let\phantomheight\vphantom
\newcommand{\lbreakh}[4][\!\!]{\left#2\vphantom{#3#4}\right.#1\allowbreaks{#3}}
\newcommand{\rbreakh}[4][\!\!]{#2#1\left.\vphantom{#2#4}\right#3}
\newcommand{\lrbreakh}[5][\!\!]{\left#2\vphantom{#3#5}\right.#1\allowbreaks{#3}#1\left.\vphantom{#3#5}\right#4}
\newcommand{\lbreak}[3][\!\!]{\lbreakh[#1]{#2}{#3}{}}
\newcommand{\rbreak}[3][\!\!]{\rbreakh[#1]{#2}{#3}{}}
\newcommand{\lrbreak}[4][\!\!]{\lrbreakh[#1]{#2}{#3}{#4}{}}
\newcommand{\olrbreak}[5][\!\!]{\overlineb{\lrbreakh[#1]{#2}{#3}{.}{#4}}{\lrbreakh[#1]{.}{#4}{#5}{#3}}}
\newcommand{\olrbreakc}[6][\!\!]{\overlineb{#2\lbreakh[#1]{#3}{#4}{#5}}{\rbreakh[#1]{#5}{#6}{#4}}}%\lrbreak[#1]{.}{#5}{#5}}}

%\DeclarePairedDelimiter\simpleset{\lbrace}{\rbrace}
\newcommand{\simpleset}[1]{\left\lbrace#1\right\rbrace}
\newcommand{\simplesetb}[1]{\lrbreak{\lbrace}{#1}{\rbrace}}

\newcommand{\mathsetb}[2][\relax]{%
\ifx#1\relax
  \simplesetb{#2}
\else
  \simplesetb{\suchthatb[#1]{#2}}
\fi}

\newcommand{\mathset}[2][\relax]{%
\ifx#1\relax
  \simpleset{#2}
\else
  \simpleset{\suchthat[#1]{#2}}
\fi}
\newcommand{\omathset}[3][\relax]{%
\ifx#1\relax
  \overlineb{\lrbreak{\lbrace}{#2}{.}}{\lrbreak{.}{#3}{\rbrace}}
\else
  \overlineb{\lrbreak{\lbrace}{\suchthat[#1]{#2}}{.}}{\lrbreak{.}{#3}{\rbrace}}
\fi}

\newcommand{\omathsetc}[4][\relax]{%
\ifx#1\relax
  \overlineb{\lrbreak{\lbrace}{\suchthat[#2]{#3}}{.}}{\lrbreak{.}{#4}{\rbrace}}
\else
  \overlineb{#1\lrbreak{\lbrace}{\suchthat[#2]{#3}}{.}}{\lrbreak{.}{#4}{\rbrace}}
\fi}
  
\newcommand{\overlinet}[2]{\overlineb{#1}{#2}}
\newcommand{\overlineb}[2]{\overline{#1\vphantom{#2}}\allowbreak\overline{#2\vphantom{#1}}}
%\newcommand{\overlinec}[3]{\overline{#1\text{\makebox[0pt]{$\phantom{#2#3}$}}}\allowbreak\overline{#2\vphantom{#1#3}}\allowbreak\overline{#3\vphantom{#1#2}}}

\newcommand{\suchthat}[2][]{#1\left\vert\vphantom{#1#2}\right.\!#2}
\newcommand{\suchthatb}[2][]{#1 \lbreakh[\!]{\vert}{#2}{#1}}
\providecommand{\scfont}{}

%texbook
%\newif\ifv@ \newif\ifh@
%\def\vphantom{\v@true\h@false\ph@nt}
%\def\hphantom{\v@false\h@true\ph@nt}
%\def\phantom{\v@true\h@true\ph@nt}
%\def\ph@nt{\ifmmode\def\next{\mathpalette\mathph@nt}%
%\else\let\next=\makeph@nt\fi \next}
%\def\makeph@nt#1{\setbox0=\hbox{#1}\finph@nt}
%\def\mathph@nt#1#2{\setbox0=\hbox{$\m@th#1{#2}$}\finph@nt}
%\def\finph@nt{\setbox2=\null \ifv@ \ht2=\ht0 \dp2=\dp0 \fi
%\ifh@ \wd2=\wd0 \fi \box2 }


%\def\m@th{\mathsurround=0pt }
\def\uncurry#1#2{#1#2} % \uncurry\macro1{{arg1}{arg2}...} -> \macro1{arg1}{arg2}...
\def\curryone#1#2#3{{\def\ndef{\noexpand\def}\def\first{\noexpand#1}\expandafter}\ifx\ndef\first #1{#2{#3}}\else #1{{#2}{#3}}\fi}
\def\currytwo#1#2#3#4{#1{{#2}{#3}{#4}}}
\def\currythree#1#2#3#4#5{#1{{#2}{#3}{#4}{#5}}}
\def\curryfour#1#2#3#4#5#6{#1{{#2}{#3}{#4}{#5}{#6}}}
\def\curryfive#1#2#3#4#5#6#7{#1{{#2}{#3}{#4}{#5}{#6}{#7}}}
\def\currysix#1#2#3#4#5#6#7#8{#1{{#2}{#3}{#4}{#5}{#6}{#7}{#8}}}
\def\curryseven#1#2#3#4#5#6#7#8#9{#1{{#2}{#3}{#4}{#5}{#6}{#7}{#8}{#9}}}


\def\uncurrytwo#1#2#3{#1#2#3}
\def\uncurriedmathpalette#1{\def\uncurried{\uncurrytwo#1}\mathpalette\uncurried}
\def\mathpalettetwo#1#2#3{\uncurriedmathpalette{#1}{{#2}{#3}}}
\def\mathpalettethree#1#2#3#4{\uncurriedmathpalette{#1}{{#2}{#3}{#4}}}


%\def\aswidthof{\v@false\h@true\@sof}
%\def\asheightof{\v@true\h@false\@sof}
%\def\assizeof{\v@true\h@true\@sof}

%\newcommand{\@sof}[3][l]{\ifmmode \def\next{\mathpalettethree\math@sof}%
%\else\let\next=\make@sof\fi \next{#1}{#2}{#3}}
%
%\def\make@sof#1#2#3{\setbox0=\hbox{#2}\setbox2=\makebox[0pt][c]{#3}\fin@sof}
%\def\math@sof#1#2#3{\setbox0=\hbox{$\m@th#1{#2}$}\setbox2=\hbox{$\m@th#1{#3}$}\fin@sof}
%\def\fin@sof{\ifv@ \ht2=\ht0 \dp2=\dp0 \fi
%\ifh@ \wd2=\wd0 \fi \box2 }
\makeatletter
\newcommand{\aswidthof}[3][c]{\ifmmode \def\next{\mathpalettethree\m@th@swidthof}%
\else\let\next=\m@ke@swidthof\fi \next{#1}{#2}{#3}}

\newdimen\@widthof
\newcommand{\m@th@swidthof}[4]{\text{\settowidth\@widthof{$\m@th#1#3$}\makebox[\@widthof][#2]{$\m@th#1#4$}}}
\newcommand{\m@ke@swidthof}[3]{\settowidth\@widthof{#2}\makebox[\@widthof][#1]{#3}}


%
%
%
%
%
%\makeatletter
%
%\def\aswidthof{\v@false\h@true\@sof}
%\def\asheightof{\v@true\h@false\@sof}
%\def\assizeof{\v@true\h@true\@sof}
%
%\def\@sof{\ifmmode\def\next##1##2{\mathpalette{\math@sof##1##2}}%
%\else\let\next=\make@sof\fi \next}
%
%\def\make@sof#1#2{\setbox0=\hbox{#1}\setbox2=\hbox{#2}\fin@sof}
%\def\math@sof#1#2#3{\setbox0=\hbox{$\m@th#3{#1}$}\setbox2=\hbox{$\m@th#3{#2}$}\fin@sof}
%\def\fin@sof{\ifv@ \ht2=\ht0 \dp2=\dp0 \fi
%\ifh@ \wd2=\wd0 \fi \box2 }

%\text{\newdimen\tempwidth \settowidth\tempwidth{$#3$}\makebox[\tempwidth][#1]{$#2$}}}% \text{\raisebox{0ex}[-\height][-\height]{$\phantom{#2}$}}}
%\newcommand{\asheightof}[3]{#1\text{\makebox[0pt]{$\phantom{#2}$}}\right.#2\left.\text{\makebox[0pt]{$\phantom{#2}$}}#3}

% 276887 sp is twice the width difference between $\left(\right)$ and $\left(\right.\left.\right)$
% \hspace{-138444 sp} \hspace{-138443 sp}\
\makeatletter
\def\breakingleft#1{{\def\templeft##1\breakingright##2{\left#1\vphantom{##1}\right.\n@space##1\n@space\left.\vphantom{##1}\right##2}\expandafter}\templeft}

\newcommand{\fullexpand}[2][]{{#1\edef\temp{#2}\expandafter}\temp}
\def\settoksexpanded#1=#2{{\edef\temp{{#2}}\expandafter}\expandafter#1\expandafter=\temp}

\newcommand{\selectfontsize}[1]{\fontsize{#1}{#1}\selectfont}

\newcommand{\newlinep}{$\left.\right.$\par\noindent}

\makeatletter
\newcommand{\interitemtext}[1]{%
\begin{list}{}
{\itemindent=0mm\labelsep=0mm
\labelwidth=0mm\leftmargin=0mm
\addtolength{\leftmargin}{-\@totalleftmargin}}
\item #1
\end{list}}
\makeatother


\newcommand{\ncr}[2]{\binom{#1}{#2}}
\newcommand{\definition}[1]{\begin{defn}#1\end{defn}}%{\par \noindent {\bf Definition:} #1}
\newcommand{\factc}[1]{\par \noindent \underline{\bf Fact:} #1}
%\providecommand{\lrang}[1]{\left\langle#1\right\rangle}%\DeclarePairedDelimiter\lrang{\langle}{\rangle}
\providecommand{\WLOG}{Without loss of generality}
\providecommand{\TFAE}{The following are equivalent}%{These facts are excellent}%(Ari)
\def\<#1>{\left\langle#1\right\rangle}
\def\[#1]{\left[#1\right]}
%\def\{#1|#2}{\mathset[#1]{#2}}
\providecommand{\End}[1]{\mathop{End}\left(#1\right)}
\providecommand{\dom}[1][\relax]{\mathop{dom}%
\ifx#1\relax%
\else%
  \left(#1\right)%
\fi}

\providecommand{\ses}{short exact sequence}
\providecommand{\st}{such that}


\newenvironment{digression}{\noindent\begin{math}%
\left(\ \begin{minipage}{0.95\textwidth}% FIX: Make this more robust
  }{\end{minipage}\ \right)\end{math}}

%\newenvironment{correction}{\textcolor{red}}{}
%======================================== Physics ===================================================
\providecommand{\uvec}[1]{{\widehat{\bf{#1}}}}
%\MakeAtOther
%\providecommand{\declareunits}[1][km,m,cm,mm,s,L,mL,kg,g]{%
%\@for\@unit:=#1\do{%
%  \declareunit{\@unit}%
%}}
%\providecommand{\declareunit}[2][\relax]{\ifx#1\relax%
%  \def\csname #2\endcsname{\text{#2}}%
%\else
%  \def\csname #1\endcsname{\text{#2}}%
%\fi}
%====================================================================================================

%================================= Proof Cases ===============================
\newcounter{ProofCasesLvlCtr}
\newcounter{ProofCasesMaxCtr}
\newcounter{ProofCasesCurCtr}
\newenvironment{proof-cases}[1][1]{%
\setcounter{ProofCasesCurCtr}{\value{ProofCasesLvlCtr}}%
\ifthenelse{\arabic{ProofCasesMaxCtr} = \arabic{ProofCasesLvlCtr}}{%
  \stepcounter{ProofCasesLvlCtr}%
  \stepcounter{ProofCasesMaxCtr}%
  \expandafter\newcounter{ProofCasesCtr\arabic{ProofCasesLvlCtr}}%
  \ifthenelse{\arabic{ProofCasesLvlCtr} = 1}{%
    \expandafter\edef\csname labelProofCases\arabic{ProofCasesLvlCtr}\endcsname{Case~\noexpand\arabic{ProofCasesCtr\arabic{ProofCasesLvlCtr}}}%
  }{%
    \expandafter\edef\csname labelProofCases\arabic{ProofCasesLvlCtr}\endcsname{\csname labelProofCases\arabic{ProofCasesCurCtr}\endcsname.\noexpand\arabic{ProofCasesCtr\arabic{ProofCasesLvlCtr}}}%
  }%
}{%
  \stepcounter{ProofCasesLvlCtr}%
  \ifthenelse{\arabic{ProofCasesLvlCtr} = 1}{%
    \expandafter\edef\csname labelProofCases\arabic{ProofCasesLvlCtr}\endcsname{Case~\noexpand\arabic{ProofCasesCtr\arabic{ProofCasesLvlCtr}}}%
  }{%
    \expandafter\edef\csname labelProofCases\arabic{ProofCasesLvlCtr}\endcsname{\csname labelProofCases\arabic{ProofCasesCurCtr}\endcsname.\noexpand\arabic{ProofCasesCtr\arabic{ProofCasesLvlCtr}}}%
  }%
}%
\begin{list}{\csname labelProofCases\arabic{ProofCasesLvlCtr}\endcsname:}{\expandafter\usecounter{ProofCasesCtr\arabic{ProofCasesLvlCtr}}}%
\expandafter\setcounter{ProofCasesCtr\arabic{ProofCasesLvlCtr}}{#1-1}%
}{\end{list}\addtocounter{ProofCasesLvlCtr}{-1}}
%\usepackage[pointlessenum]{paralist}
%\newenvironment{proof-cases}{\begin{enumerate}[{Case} 1:]}{\end{enumerate}}
%=============================== End Proof Cases =============================

\allowdisplaybreaks[1]

\usepackage[margin=1in]{geometry}

\usepackage{cancel}
% email to abhinav@math.mit.edu
\def\d{\, {\rm d}}

%\usepackage{pgf,tikz}
%\usetikzlibrary{arrows}
\usepackage{wasysym}
\usepackage{pdfcomment}

\usepackage{datetime}
\usepackage{verbatim}
\usepackage[all]{xy}
\def\classnumber{18.721}
\def\classname{Algebraic Geometry}
\edef\cheadcontent{\classnumber\space Notes}

\pagestyle{fancy}
\headheight 13.7pt
\fancyhead{}
\fancyhead{Jason Gross}%\today}
\fancyfoot{}
\lhead{Jason Gross}
\rhead{\TeX ed on \today}
\chead{\cheadcontent}
\cfoot{\thepage\space of \pageref{LastPage}}

\newcommand{\flag}[2][]{#2\footnote{#1}}

\begin {document}
\setcounter {section}{14}\section {Wednesday, March 9, 2011}
  %\subsection*{Corrections from Last Time}
  \begin{thm*}
    Let $A$ be a finite type algebra and a domain, and let $K$ be the field of fractions.  Then the integral closure of $A$ is a finite extension $L / K$, and is a finite $A$-module.
  \end{thm*}
  \begin{proof}
    Had the trace pairing $L \times L \to K$, $x, y \leadsto tr(xy)$.  It's non-degenerate because:

    $y x \ne 0$, put $y = x^{-1}$

    $\langle x, y \rangle = t(1) = [L:K]$

    Reduce to the case of $A$ integrally closed by Noether Normalization.

    $k[y] \subset A$, $A$ a finite $k[y]$-module

    $k[y] \subset K \subset L$.

    Replace $A$ by $k[y]$.

    $\therefore$, we may assume that $A$ is integrally closed.  Then if $\alpha \in L$ is integral over $A$, $tr(\alpha) \in A$.

    Therefore, if $B$ is any subring of $L$, and a finite $A$-module, then all elements are integral over $A$.  Therefore, $B\times B \to A$, $x, y \leadsto \langle x, y\rangle = tr(xy)$.

    \emph{We want to show} that there is a maximal such $B$.  Then $B$ is the integral closure in $L$.

    Start with one, $B_0$, big enough so that it contains a basis for $L / K$.  (We can do this because for any $\gamma\in L$, $\gamma$ is algebraic in $K$; $\gamma^n - a_1 \gamma^{n-1} + \cdots \pm a_n = 0$ with $a_i\in K$.  Since $K$ is the field of fractions, we can clear denominators, getting $d\gamma^n - a_1' \gamma^{n-1} + \cdots \pm a_n' = 0$, with $d, a_i'\in A$.  Then $d\gamma$ is integral over $A$; multiply everything by $d^{n-1}$.)  Denote the basis by $(v_1, \ldots, v_n)$, $v_i\in B_0$, $n = [L:K]$.

    Investigate some larger algebra (which is still a finite $A$-module) $B$.  Then $A\subset B_0 \subset B$.
    \begin{align*}
      B \times B & \to A \\
      \beta, v_i & \leadsto b_i \defeq \langle \beta, v_i\rangle\in A \\
      \intertext{map}
      \beta & \leadsto (b_1, \ldots, b_n)\in A^n \\
      B & \stackrel{\Phi}{\to} A^n
    \end{align*}
    This is $A$-linear (homomorphism of $A$-modules)

    $\Phi$ is injective: $\Phi(\beta) = 0$ means $\langle \beta, v_i\rangle = 0$ for all $i$.  Since $\{v_i\}$ is a basis for $L / K$, $\langle \beta, y \rangle = 0$ for all $y \in L$.  Thus, $\beta = 0$.

    Then we can identify $B$ with $\Phi(B)$ as a submodule of $A^n$.  Since $A$ is noetherian, submodules have the ascending chain condition.
  \end{proof}

  (DIGRESSION ABOUT GALOIS GROUPS AND $G$-ORIBTS NOT INCLUDED)

  Let $A \subset B$ be a domain, $A$ finite-type and integrally closed, and $B$ a finite $A$-module.

  What about prime ideals in $A$ and $B$?
  \begin{itemize}
    \item[] extended ideal of $P\subset A$ is $P^e = PB$ ideal of $B$
    \item[] contracted ideal of $Q\subset B$ is $Q^c = A \cap Q$ ideal of $A$
  \end{itemize}

  \begin{fact*}[General Fact]
    If $Q$ is a prime ideal of $B$, then $Q^c$ is a prime ideal of $A$.
    $$
    \xymatrix{
      A \ar@{^{(}->}[rr] \ar[dr]_{\text{kernel $Q^c$}} && B \ar[dl]^{\text{kernel $Q$}} \\
      & B / Q \text{\makebox[0pt][l]{a domain ($Q$ prime)}}
    }
    $$
    The image is a domain, so $Q^c$ is a prime ideal.
  \end{fact*}

  Back to the case above,
  $$
  \xymatrix @R=0.5em{
    A \ar@{^{(}->}[r] & B \\
    P' & Q' & \text{we mean }P' = A \cap Q' \\
    P & Q & P = A \cap Q,\quad Q'\supset Q
  }
  $$

  \begin{fact}[Lying Over]
    Given $P$ a prime ideal of $A$, there exists a $A$ prime ideal of $B$, with $A \cap Q = P$.  (The map $\pspec B \to \pspec A$ is injective.)
  \end{fact}

  \begin{fact}[Going Up]
    Given
    $$
    \xymatrix @R=0.25em{
      A \ar@{^{(}->}[r] & B \\
      P' &  *+[o][F.]{Q'} & \ar[l] \text{exists} \\
      P & Q
    }
    $$
  \end{fact}

  \begin{fact}[Going Down]
    If $A$ is integrally closed, then given
    $$
    \xymatrix @R=0.25em{
      A \ar@{^{(}->}[r] & B \\
      P' &  Q' \\
      P & *+[o][F.]{Q} & \ar[l] \text{exists}
    }
    $$
  \end{fact}

  \begin{lem*}
    Given
    $$
    \xymatrix @R=0.25em{
      A \ar@{^{(}->}[r] & B \\
      P' &  Q' \\
      P & Q
    }
    $$
    If $P' = P$, then $Q' = Q$.
  \end{lem*}
  \begin{proof}
    \begin{enumerate}[{Case }1:]
      \item $P = 0$.  We need to show that if $P' = 0$, then $Q' = 0$.  Take $\alpha \in Q' \subset B$.  Then $\alpha$ is integral over $A$ (because it's in $B$?).  We have $\alpha^r - a_1 \alpha^{r-1} + \cdots \pm a_r = 0$.  Since $a_r \in Q'$ and $a_r\in A$, we have $a_r \in P' = \{0\}$.  Thus $a_r = 0$.  If $\alpha \ne 0$, then cancel $\alpha$ from the relation, and repeat until $r = 1$.  Then $\alpha^1 - a_1 = 0$.  Since $a_1 \in A$ and $a_1 \in Q'$, $a_1\in P'$.  Then $a_1 = 0$ and $\alpha = 0$, which is a contradiction.
      \item (general case).  Go to $\bar A = A / P \subset \bar B = B / Q$.
        $$
        \xymatrix @R=0.25em{
          \bar A \ar@{^{(}->}[r] & \bar B \\
          \bar P' &  \bar Q' \\
          (0) & (0)
        }
        $$
        Then case 1 says that $\bar P' = (0)$ implies that $\bar Q' = (0)$.  Then $\bar P' = (0) \iff P' = P$ and $\bar Q' = (0) \iff Q' = Q$.
    \end{enumerate}
  \end{proof}
  \begin{lem*}
    If $Q$ is a maximal ideal of $B$, then $Q^c = P$ is maximal in $A$.
  \end{lem*}
  \begin{proof}
    $Q$ is maximal in $B$ if and only if $\bar B = B / Q$ is a field.  Then $A / P = \bar A \subset \bar B$ is a field.  Then $\bar B$ is a finite $\bar A$-module.
    \begin{lem*}
      If $A \subset B$ is a field and $B$ is a finite $A$-module, then $A$ is a field.
    \end{lem*}
    \begin{proof}
      Take a non-zero element in $\alpha\in A$ (we want to show that it's invertible).  Then $\alpha$ is invertible in $B$.  Since $B$ is a finite $A$-module, so $u = \alpha^{-1}$ is integral over $A$.  Then $u^r - a_1 u^{r-1} + \cdots \pm a_r = 0$ with $a_i \in A$.  Multiply by $\alpha^{r - 1}$.  Then we get $u - a_1 + a_2 \alpha - \cdots \pm a_r \alpha^r = 0$, with all of these elements of $A$.
    \end{proof}
    (To be finished next time?)
  \end{proof}
\end {document}

