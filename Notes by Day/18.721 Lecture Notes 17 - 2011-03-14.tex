\documentclass [letterpaper,11pt,twoside]{article}
%\usepackage{pst-pdf,pst-text,pstricks-add}
\providecommand{\ifincludeall}{\iffalse}
\usepackage{fancyhdr}
\usepackage{lastpage}
\usepackage{enumerate}
\ifincludeall

  \usepackage{wrapfig}

%================================ Unicode ================================
  \usepackage[utf8]{inputenc}
  \DeclareUnicodeCharacter{916}{\ensuremath{\Delta}}
  \DeclareUnicodeCharacter{937}{\ensuremath{\Omega}}
  \DeclareUnicodeCharacter{949}{\ensuremath{\epsilon}}
  \DeclareUnicodeCharacter{956}{\ensuremath{\mu}}
  \DeclareUnicodeCharacter{963}{\ensuremath{\sigma}}
  \DeclareUnicodeCharacter{977}{\ensuremath{\theta}}
  \DeclareUnicodeCharacter{1009}{\ensuremath{\rho}}
  \DeclareUnicodeCharacter{03B4}{\ensuremath{\delta}}
  \DeclareUnicodeCharacter{221A}{\sqrt}
  \DeclareUnicodeCharacter{2124}{\ensuremath{\mathbb Z}}
%============================== End Unicode ==============================
\fi


\usepackage{amsmath}
\usepackage{amssymb}
%================================= AMSTHM =================================
\usepackage{amsthm}

\newtheorem{thm}{Theorem}[section]
%\newtheorem{theorem}{Theorem}
\newtheorem{conjecture}{Conjecture}[section]
\newtheorem{lem}{Lemma}[section]
%\newtheorem{lemma}[theorem]{Lemma}
\newtheorem{cor}[thm]{Corollary}
%\newtheorem{corollary}[theorem]{Corollary}
\newtheorem{prop}[thm]{Proposition}
%\newtheorem{proposition}[theorem]{Proposition}
%\newtheorem{definition}[theorem]{Definition}
%\newtheorem{example}[theorem]{Example}
\newtheorem{exercise}[thm]{Exercise}
%\newtheorem{exercise}[theorem]{Exercise}
\newtheorem{claim}[thm]{Claim}
\newtheorem{law}{Law}[section]
\newtheorem*{thm*}{Theorem}
\newtheorem*{lem*}{Lemma}
\newtheorem*{conjecture*}{Conjecture}
\newtheorem*{cor*}{Corollary}
\newtheorem*{prop*}{Proposition}
\newtheorem*{exercise*}{Exercise}
\newtheorem*{law*}{Law}
\newtheorem*{claim*}{Claim}



\theoremstyle{definition} \newtheorem{defn}{Definition}[section]
\theoremstyle{definition} \newtheorem*{defn*}{Definition}
\newtheorem{example}[thm]{Example}
\newtheorem*{example*}{Example}
\newtheorem{eg}[thm]{Example}
\newtheorem*{eg*}{Example}

\newtheorem{fact}{Fact}[section]
\newtheorem*{fact*}{Fact}

\newcommand{\thmref}[1]{Theorem~\ref{#1}}
%=============================== End AMSTHM ===============================
\usepackage{esint}
\ifincludeall
  \usepackage[table]{xcolor}
  \usepackage{xifthen}
\fi
%\if dcpic
%\usepackage{dcpic}
%\else

%\ifincludeall
  \usepackage{ifpdf}
%\else
  %\newif\ifpdf
  %\pdftrue
%\fi

\ifincludeall
  \ifpdf
    \usepackage[pdftex]{graphicx}
    \usepackage{pdfpages}
    \usepackage[plainpages=false,pdfpagelabels,unicode]{hyperref}
  \else
    \usepackage[dvips]{graphicx}
    \usepackage{pstricks,pstricks-add,pst-math,pst-xkey}
  \fi
\else
  \usepackage[pdftex]{graphicx}
  \usepackage[pdfpagelabels,unicode]{hyperref}
\fi

%\ifincludeall
  \usepackage{mathtools}
%\fi

%\global\def\isxy{}
%\global\let\isxy\relax
\ifincludeall
  \providecommand{\isxy}{\let\isxy\relax}
  \expandafter\ifx\isxy\relax
    %\usepackage{mathpazo} or \usepackage{mathptmx} 
    \usepackage{flexisym} % flexisym package is required by breqn
                          % can load as \usepackage[mathpazo]{flexisym}
    \usepackage{breqn}
  \else
    \usepackage[all]{xy}
  \fi
\fi



%\usepackage[exponent-product=\cdot,per-mode=fraction,quotient-mode=fraction,fraction-function=\sfrac]{siunitx} %alsoload={named,prefixed,abbr,hep},
%\newunit{\statvolt}{statV}
%\newunit{\erg}{erg}
%\newunit{\esu}{esu}

\providecommand{\abs}[1]{\left\lvert#1\right\rvert}%\DeclarePairedDelimiter\abs{\lvert}{\rvert} %\providecommand{\abs}[1]{\lvert#1\rvert}
\ifincludeall
  \DeclarePairedDelimiter\norm{\lVert}{\rVert}
  \DeclarePairedDelimiter\floor{\lfloor}{\rfloor}
\else
  \providecommand{\floor}[1]{\left\lfloor #1\right\rfloor}
  \providecommand{\norm}[1]{\lVert#1\rVert}
\fi
\newcommand{\gcdf}[2]{\left( #1 , #2 \right)}
\DeclareMathOperator{\lcm}{lcm}
\DeclareMathOperator{\im}{im}
\DeclareMathOperator{\rank}{rank}
\DeclareMathOperator{\spans}{span}
\DeclareMathOperator{\divergence}{div}
\DeclareMathOperator{\tr}{tr}
\DeclareMathOperator{\grad}{grad}
\DeclareMathOperator{\spec}{Spec}
\DeclareMathOperator{\pspec}{PSpec}
\DeclareMathOperator{\rad}{rad}
\DeclareMathOperator{\trdeg}{tr\,deg}
\DeclareMathOperator{\gk}{gk}
\DeclareMathOperator{\fract}{fract}
\newcommand{\lcmf}[2]{\lcm\left( #1 , #2 \right)}
\def\dbar{{\mathchar'26\mkern-12mu d}}

\def\sfrac#1/#2{\leavevmode\kern.1em\raise.5ex\hbox{\the\scriptfont0 #1}\kern-.1em/\kern-.15em\lower.25ex\hbox{\the\scriptfont0 #2}}

\renewcommand{\d}{\,d}

\newcommand{\defeq}{\coloneqq}%\stackrel{\mathrm{df}}{=}}%{\ensuremath{:=}}%

\ifincludeall
  \newcommand{\complementset}[1][\ \ \rule{0pt}{1ex}]{\overline{#1}}
  \newcommand{\boldcomplementset}[1][\ \ \rule{0pt}{1ex}]{\text{\makebox[0pt][l]{$#1$}}\rule[\heightof{#1}+3pt]{\widthof{#1}}{0.11ex}}
  \let\oldboldsymbol=\boldsymbol
  \renewcommand{\boldsymbol}[1]{\let\oldcomplementset=\complementset%
  %\oldboldsymbol{#1}\  	%
  \let\complementset\boldcomplementset%
  \oldboldsymbol{#1}%
  \let\complementset=\oldcomplementset}
\fi
% \gdef\phantomhdepth{\relax}
% \gdef\phantomhdepth{1}
% \newcommand{\phantomheight}[1]{         %
% \ifx\phantomhdepth\relax          %
%   \let\phantomhdepth{1}          %
%   \vphantom{#1}          %
%   \let\phantomhdepth\relax          %
% \fi          %
% }
\newcommand{\tuple}[1]{\breakingtuple{#1}}
\newcommand{\nbtuple}[1]{\left(#1\right)}
\newcommand{\breakingtuple}[1]{\lrbreak{(}{#1}{)}}         %$
         %\let\oldcomma=,
\begingroup
  \lccode`~=`,
  \lowercase{\endgroup
    \let\oldcomma=~
    \def\comma{\oldcomma}
    \def~{\comma}         %
  }         %
         %\def\aaa{\comma}
         %\mathcode`,="613B

\newcommand{\allowbreaks}[1]{\begingroup \mathcode`,="8000 \def\comma{\oldcomma\allowbreak}#1\def\comma{\oldcomma} \mathcode`,="613B \endgroup }         %\replace{#1}{,}{,\allowbreak}}
%\let\phantomheight\vphantom
\newcommand{\lbreakh}[4][\!\!]{\left#2\vphantom{#3#4}\right.#1\allowbreaks{#3}}
\newcommand{\rbreakh}[4][\!\!]{#2#1\left.\vphantom{#2#4}\right#3}
\newcommand{\lrbreakh}[5][\!\!]{\left#2\vphantom{#3#5}\right.#1\allowbreaks{#3}#1\left.\vphantom{#3#5}\right#4}
\newcommand{\lbreak}[3][\!\!]{\lbreakh[#1]{#2}{#3}{}}
\newcommand{\rbreak}[3][\!\!]{\rbreakh[#1]{#2}{#3}{}}
\newcommand{\lrbreak}[4][\!\!]{\lrbreakh[#1]{#2}{#3}{#4}{}}
\newcommand{\olrbreak}[5][\!\!]{\overlineb{\lrbreakh[#1]{#2}{#3}{.}{#4}}{\lrbreakh[#1]{.}{#4}{#5}{#3}}}
\newcommand{\olrbreakc}[6][\!\!]{\overlineb{#2\lbreakh[#1]{#3}{#4}{#5}}{\rbreakh[#1]{#5}{#6}{#4}}}%\lrbreak[#1]{.}{#5}{#5}}}

%\DeclarePairedDelimiter\simpleset{\lbrace}{\rbrace}
\newcommand{\simpleset}[1]{\left\lbrace#1\right\rbrace}
\newcommand{\simplesetb}[1]{\lrbreak{\lbrace}{#1}{\rbrace}}

\newcommand{\mathsetb}[2][\relax]{%
\ifx#1\relax
  \simplesetb{#2}
\else
  \simplesetb{\suchthatb[#1]{#2}}
\fi}

\newcommand{\mathset}[2][\relax]{%
\ifx#1\relax
  \simpleset{#2}
\else
  \simpleset{\suchthat[#1]{#2}}
\fi}
\newcommand{\omathset}[3][\relax]{%
\ifx#1\relax
  \overlineb{\lrbreak{\lbrace}{#2}{.}}{\lrbreak{.}{#3}{\rbrace}}
\else
  \overlineb{\lrbreak{\lbrace}{\suchthat[#1]{#2}}{.}}{\lrbreak{.}{#3}{\rbrace}}
\fi}

\newcommand{\omathsetc}[4][\relax]{%
\ifx#1\relax
  \overlineb{\lrbreak{\lbrace}{\suchthat[#2]{#3}}{.}}{\lrbreak{.}{#4}{\rbrace}}
\else
  \overlineb{#1\lrbreak{\lbrace}{\suchthat[#2]{#3}}{.}}{\lrbreak{.}{#4}{\rbrace}}
\fi}
  
\newcommand{\overlinet}[2]{\overlineb{#1}{#2}}
\newcommand{\overlineb}[2]{\overline{#1\vphantom{#2}}\allowbreak\overline{#2\vphantom{#1}}}
%\newcommand{\overlinec}[3]{\overline{#1\text{\makebox[0pt]{$\phantom{#2#3}$}}}\allowbreak\overline{#2\vphantom{#1#3}}\allowbreak\overline{#3\vphantom{#1#2}}}

\newcommand{\suchthat}[2][]{#1\left\vert\vphantom{#1#2}\right.\!#2}
\newcommand{\suchthatb}[2][]{#1 \lbreakh[\!]{\vert}{#2}{#1}}
\providecommand{\scfont}{}

%texbook
%\newif\ifv@ \newif\ifh@
%\def\vphantom{\v@true\h@false\ph@nt}
%\def\hphantom{\v@false\h@true\ph@nt}
%\def\phantom{\v@true\h@true\ph@nt}
%\def\ph@nt{\ifmmode\def\next{\mathpalette\mathph@nt}%
%\else\let\next=\makeph@nt\fi \next}
%\def\makeph@nt#1{\setbox0=\hbox{#1}\finph@nt}
%\def\mathph@nt#1#2{\setbox0=\hbox{$\m@th#1{#2}$}\finph@nt}
%\def\finph@nt{\setbox2=\null \ifv@ \ht2=\ht0 \dp2=\dp0 \fi
%\ifh@ \wd2=\wd0 \fi \box2 }


%\def\m@th{\mathsurround=0pt }
\def\uncurry#1#2{#1#2} % \uncurry\macro1{{arg1}{arg2}...} -> \macro1{arg1}{arg2}...
\def\curryone#1#2#3{{\def\ndef{\noexpand\def}\def\first{\noexpand#1}\expandafter}\ifx\ndef\first #1{#2{#3}}\else #1{{#2}{#3}}\fi}
\def\currytwo#1#2#3#4{#1{{#2}{#3}{#4}}}
\def\currythree#1#2#3#4#5{#1{{#2}{#3}{#4}{#5}}}
\def\curryfour#1#2#3#4#5#6{#1{{#2}{#3}{#4}{#5}{#6}}}
\def\curryfive#1#2#3#4#5#6#7{#1{{#2}{#3}{#4}{#5}{#6}{#7}}}
\def\currysix#1#2#3#4#5#6#7#8{#1{{#2}{#3}{#4}{#5}{#6}{#7}{#8}}}
\def\curryseven#1#2#3#4#5#6#7#8#9{#1{{#2}{#3}{#4}{#5}{#6}{#7}{#8}{#9}}}


\def\uncurrytwo#1#2#3{#1#2#3}
\def\uncurriedmathpalette#1{\def\uncurried{\uncurrytwo#1}\mathpalette\uncurried}
\def\mathpalettetwo#1#2#3{\uncurriedmathpalette{#1}{{#2}{#3}}}
\def\mathpalettethree#1#2#3#4{\uncurriedmathpalette{#1}{{#2}{#3}{#4}}}


%\def\aswidthof{\v@false\h@true\@sof}
%\def\asheightof{\v@true\h@false\@sof}
%\def\assizeof{\v@true\h@true\@sof}

%\newcommand{\@sof}[3][l]{\ifmmode \def\next{\mathpalettethree\math@sof}%
%\else\let\next=\make@sof\fi \next{#1}{#2}{#3}}
%
%\def\make@sof#1#2#3{\setbox0=\hbox{#2}\setbox2=\makebox[0pt][c]{#3}\fin@sof}
%\def\math@sof#1#2#3{\setbox0=\hbox{$\m@th#1{#2}$}\setbox2=\hbox{$\m@th#1{#3}$}\fin@sof}
%\def\fin@sof{\ifv@ \ht2=\ht0 \dp2=\dp0 \fi
%\ifh@ \wd2=\wd0 \fi \box2 }
\makeatletter
\newcommand{\aswidthof}[3][c]{\ifmmode \def\next{\mathpalettethree\m@th@swidthof}%
\else\let\next=\m@ke@swidthof\fi \next{#1}{#2}{#3}}

\newdimen\@widthof
\newcommand{\m@th@swidthof}[4]{\text{\settowidth\@widthof{$\m@th#1#3$}\makebox[\@widthof][#2]{$\m@th#1#4$}}}
\newcommand{\m@ke@swidthof}[3]{\settowidth\@widthof{#2}\makebox[\@widthof][#1]{#3}}


%
%
%
%
%
%\makeatletter
%
%\def\aswidthof{\v@false\h@true\@sof}
%\def\asheightof{\v@true\h@false\@sof}
%\def\assizeof{\v@true\h@true\@sof}
%
%\def\@sof{\ifmmode\def\next##1##2{\mathpalette{\math@sof##1##2}}%
%\else\let\next=\make@sof\fi \next}
%
%\def\make@sof#1#2{\setbox0=\hbox{#1}\setbox2=\hbox{#2}\fin@sof}
%\def\math@sof#1#2#3{\setbox0=\hbox{$\m@th#3{#1}$}\setbox2=\hbox{$\m@th#3{#2}$}\fin@sof}
%\def\fin@sof{\ifv@ \ht2=\ht0 \dp2=\dp0 \fi
%\ifh@ \wd2=\wd0 \fi \box2 }

%\text{\newdimen\tempwidth \settowidth\tempwidth{$#3$}\makebox[\tempwidth][#1]{$#2$}}}% \text{\raisebox{0ex}[-\height][-\height]{$\phantom{#2}$}}}
%\newcommand{\asheightof}[3]{#1\text{\makebox[0pt]{$\phantom{#2}$}}\right.#2\left.\text{\makebox[0pt]{$\phantom{#2}$}}#3}

% 276887 sp is twice the width difference between $\left(\right)$ and $\left(\right.\left.\right)$
% \hspace{-138444 sp} \hspace{-138443 sp}\
\makeatletter
\def\breakingleft#1{{\def\templeft##1\breakingright##2{\left#1\vphantom{##1}\right.\n@space##1\n@space\left.\vphantom{##1}\right##2}\expandafter}\templeft}

\newcommand{\fullexpand}[2][]{{#1\edef\temp{#2}\expandafter}\temp}
\def\settoksexpanded#1=#2{{\edef\temp{{#2}}\expandafter}\expandafter#1\expandafter=\temp}

\newcommand{\selectfontsize}[1]{\fontsize{#1}{#1}\selectfont}

\newcommand{\newlinep}{$\left.\right.$\par\noindent}

\makeatletter
\newcommand{\interitemtext}[1]{%
\begin{list}{}
{\itemindent=0mm\labelsep=0mm
\labelwidth=0mm\leftmargin=0mm
\addtolength{\leftmargin}{-\@totalleftmargin}}
\item #1
\end{list}}
\makeatother


\newcommand{\ncr}[2]{\binom{#1}{#2}}
\newcommand{\definition}[1]{\begin{defn}#1\end{defn}}%{\par \noindent {\bf Definition:} #1}
\newcommand{\factc}[1]{\par \noindent \underline{\bf Fact:} #1}
%\providecommand{\lrang}[1]{\left\langle#1\right\rangle}%\DeclarePairedDelimiter\lrang{\langle}{\rangle}
\providecommand{\WLOG}{Without loss of generality}
\providecommand{\TFAE}{The following are equivalent}%{These facts are excellent}%(Ari)
\def\<#1>{\left\langle#1\right\rangle}
\def\[#1]{\left[#1\right]}
%\def\{#1|#2}{\mathset[#1]{#2}}
\providecommand{\End}[1]{\mathop{End}\left(#1\right)}
\providecommand{\dom}[1][\relax]{\mathop{dom}%
\ifx#1\relax%
\else%
  \left(#1\right)%
\fi}

\providecommand{\ses}{short exact sequence}
\providecommand{\st}{such that}


\newenvironment{digression}{\noindent\begin{math}%
\left(\ \begin{minipage}{0.95\textwidth}% FIX: Make this more robust
  }{\end{minipage}\ \right)\end{math}}

%\newenvironment{correction}{\textcolor{red}}{}
%======================================== Physics ===================================================
\providecommand{\uvec}[1]{{\widehat{\bf{#1}}}}
%\MakeAtOther
%\providecommand{\declareunits}[1][km,m,cm,mm,s,L,mL,kg,g]{%
%\@for\@unit:=#1\do{%
%  \declareunit{\@unit}%
%}}
%\providecommand{\declareunit}[2][\relax]{\ifx#1\relax%
%  \def\csname #2\endcsname{\text{#2}}%
%\else
%  \def\csname #1\endcsname{\text{#2}}%
%\fi}
%====================================================================================================

%================================= Proof Cases ===============================
\newcounter{ProofCasesLvlCtr}
\newcounter{ProofCasesMaxCtr}
\newcounter{ProofCasesCurCtr}
\newenvironment{proof-cases}[1][1]{%
\setcounter{ProofCasesCurCtr}{\value{ProofCasesLvlCtr}}%
\ifthenelse{\arabic{ProofCasesMaxCtr} = \arabic{ProofCasesLvlCtr}}{%
  \stepcounter{ProofCasesLvlCtr}%
  \stepcounter{ProofCasesMaxCtr}%
  \expandafter\newcounter{ProofCasesCtr\arabic{ProofCasesLvlCtr}}%
  \ifthenelse{\arabic{ProofCasesLvlCtr} = 1}{%
    \expandafter\edef\csname labelProofCases\arabic{ProofCasesLvlCtr}\endcsname{Case~\noexpand\arabic{ProofCasesCtr\arabic{ProofCasesLvlCtr}}}%
  }{%
    \expandafter\edef\csname labelProofCases\arabic{ProofCasesLvlCtr}\endcsname{\csname labelProofCases\arabic{ProofCasesCurCtr}\endcsname.\noexpand\arabic{ProofCasesCtr\arabic{ProofCasesLvlCtr}}}%
  }%
}{%
  \stepcounter{ProofCasesLvlCtr}%
  \ifthenelse{\arabic{ProofCasesLvlCtr} = 1}{%
    \expandafter\edef\csname labelProofCases\arabic{ProofCasesLvlCtr}\endcsname{Case~\noexpand\arabic{ProofCasesCtr\arabic{ProofCasesLvlCtr}}}%
  }{%
    \expandafter\edef\csname labelProofCases\arabic{ProofCasesLvlCtr}\endcsname{\csname labelProofCases\arabic{ProofCasesCurCtr}\endcsname.\noexpand\arabic{ProofCasesCtr\arabic{ProofCasesLvlCtr}}}%
  }%
}%
\begin{list}{\csname labelProofCases\arabic{ProofCasesLvlCtr}\endcsname:}{\expandafter\usecounter{ProofCasesCtr\arabic{ProofCasesLvlCtr}}}%
\expandafter\setcounter{ProofCasesCtr\arabic{ProofCasesLvlCtr}}{#1-1}%
}{\end{list}\addtocounter{ProofCasesLvlCtr}{-1}}
%\usepackage[pointlessenum]{paralist}
%\newenvironment{proof-cases}{\begin{enumerate}[{Case} 1:]}{\end{enumerate}}
%=============================== End Proof Cases =============================

\allowdisplaybreaks[1]

\usepackage[margin=1in]{geometry}

\usepackage{cancel}
% email to abhinav@math.mit.edu
\def\d{\, {\rm d}}

%\usepackage{pgf,tikz}
%\usetikzlibrary{arrows}
\usepackage{wasysym}
\usepackage{pdfcomment}

\usepackage{datetime}
\usepackage{verbatim}
\usepackage[all]{xy}
\def\classnumber{18.721}
\def\classname{Algebraic Geometry}
\edef\cheadcontent{\classnumber\space Notes}

\pagestyle{fancy}
\headheight 13.7pt
\fancyhead{}
\fancyhead{Jason Gross}%\today}
\fancyfoot{}
\lhead{Jason Gross}
\rhead{\TeX ed on \today}
\chead{\cheadcontent}
\cfoot{\thepage\space of \pageref{LastPage}}

\newcommand{\flag}[2][]{#2\footnote{#1}}

\begin {document}
\setcounter {section}{16}\section {Monday, March 14, 2011}
  \begin{defn*}
    The \emph{Krull dimension} of a ring $A$ is the length of the longest chain of prime ideals $P_0 < P_1 < \cdots < P_d$.
  \end{defn*}
  In a dimension zero ring, every prime ideal is maximal and also minimal.  Therefore, there are a finite number of prime/minimal ideals.

  If $A$ is a domain of dimension 1, then $(0)$ is a prime ideal and all the other prime ideals are maximal.

  \begin{defn*}
    The \emph{codimension} of a prime ideal $P$ is the length of the longest chain $P_0 < P_1 < \cdots < P_d = P$.
  \end{defn*}
  If $A$ is a domain, then $P$ has codimension 0 if $P = (0)$, and $P$ has codimension 1 if it's not zero and there does not exist $P'$ with $(0) < P' < P$.

  \begin{thm*}[Knull's Principal Ideal Theorem]
    Let $A$ be a domain.  Let $x\in A$ with $x\ne 0$, and let $P$ be a prime ideal.  If $x\in P$ and $P$ is is the minimal prime ideal containing $x$, then $P$ has codimension 1.

    ($P /  x\in P \longleftrightarrow \text{prime ideals }\bar B\text{ of }\bar A = A / (x)$.

    Therefore, the Krull dimension of $\bar A$ is the Krull dimension of $A$, - 1.)
  \end{thm*}
  \subsection*{Discrete valuations}
    Generalize order of vanishing of $f(x)$ at $x = a$.  For $f(x_1, \ldots, x_n)$, talking about order of vanishing at $(a_1, \ldots, a_n)$ doesn't make much sense.  But we can define the order of vanishing along a subvariety of codimension 1.

    Let $K$ be a field.  A discrete valuation(?) $v$ is a group homomorphism $K^\times \stackrel{v}{\to} \mathbb Z^+$ such that $v(\alpha + \beta) \ge \min\{v(\alpha), v(\beta)\}$ and $v(\alpha\beta) = v(\alpha) + v(\beta)$.  ``$v(\alpha) = k$ means $\alpha$ vanishes to order $k$ (has a zero of order $k$)''.  $v(\alpha) = -k$, ``$\alpha$ has a \emph{pole} of order $k$.

    A homomorphism $K^\times \stackrel{v}{\to}\Gamma^+$ (an ordered group with $+$) is also a valuation.  This is unimportant!

    \begin{defn*}
      The (discrete) valuation ring $R$ associated to a discrete valuation $v$ is $R = \mathset[\alpha\in K^\times]{v(\alpha) \ge 0} \cup \{0\}$ (``no pole'').
    \end{defn*}
    Remorse: You always have to add $\{0\}$, or define $v(0) = \infty$ (but this is artificial)

    \begin{defn*}
      A \emph{fractional ideal} is a non-zero finitely generated submodule of $K$.
    \end{defn*}

    Properties of DVR (discrete valuation ring):
    \begin{itemize}
      \item subring of $K$ and local domain (local ring that's a domain, ring with one maximal ideal)
      \item $\mathfrak M = \mathset[\alpha\in K]{v(\alpha) > 0} \cup \{0\}$
      \item $\mathfrak M$ is a principle ideal generated by any $t\in K^\times$ with $v(t) = 1$.
      \item ideals of $R$ are $\mathfrak M^k = (t^k)$ and zero ideal ($K \ge 0$)  ($\mathfrak M^k = \mathset[\alpha]{v(\alpha) \ge k}$)
      \item $R$ is normal
      \item The fractional ideals of $R$ are $(t^k)$, $k\in \mathbb Z$.
    \end{itemize}
    Write $(t^k)  = \mathfrak M^k$ also for $k < 0$.  Then $\mathfrak M^{k+l} = \mathfrak M^k \mathfrak M^l$, $k,l \in \mathbb Z$.

    \begin{lem*}
      $v(1) = 0$, $v(-1) = 0$, $v(a^{-1}) = -v(a)$
    \end{lem*}
    Take $\alpha\in K^\times$, $v(\alpha) = k$.  Also, $v(t^k) = k$.  $v(t^{-k}\alpha) = 0$. Then $t^{-k}\alpha = u$ is a unit in $R$, so $\alpha = u t^k$.

    \begin{prop*}
      The following are equivalent conditions on a local noetherian domain $A$:
      \begin{enumerate}[(1)]
        \item $A$ is a DVR
        \item $A$ is normal and has dimension 1
        \item $A$ is normal and there exists an $x\in A$ such that $\mathfrak M$ is the minimal prime ideal containing $x$
        \item $\mathfrak M$ is principle
      \end{enumerate}
    \end{prop*}
    \begin{proof} \par\noindent
      \begin{itemize}
        \item[$(1)\implies (2)$] We're not doing it (it follows from the properties of a DVR)
        \item[$(2) \implies (3)$] There exist only two prime ideals in $A$: $(0) \ne \mathfrak M$.  $x\in \mathfrak M$, $x\ne 0$ also works. %Take $x \ne 0$ in $\mathfrak M$
        \item[$(3) \implies (4)$] Take $x$ as in $(3)$.  Then $\bar A = A / (x)$ has only one prime ideal $\bar{\mathfrak M}$ both maximal and minimal.  The intersection of the minimal prime ideals is the nilradical of $\bar A$.  Therefore $\bar{\mathfrak M}$ is the nilradical.  Therefore, $\bar{\mathfrak M}^N = (0)$, $N \gg 0$.  Thus $\mathfrak M^N < (x)$.

          Choose $r$ such that $\mathfrak M^{r-1} \not\subset (x)$ but $\mathfrak M^r \subset (x)$.  Take $y\in \mathfrak M^{r-1}$, $y\notin (x)$.  We want to show that $w = x / y$ generates $\mathfrak M$.  Let $z = w^{-1} = y / x$.  Since $y\notin (x)$, $z\notin A$.  Now consider $z\mathfrak M$.
          \begin{lem*}
            Let $A$ be a normal noetherian domain, $I$ a non-zero ideal, $\gamma \in K = \operatorname{Fract}(A)$.  If $\gamma I \subset I$, then $\gamma \in A$.
          \end{lem*}
          \begin{proof}
            $\gamma I \subset I$ means that $I$ is an $A[\gamma]$-module.  $I$ ($\gamma I$?) is faithful, and a finite $A$-module.  Then $\gamma$ is integral over $A$, and thus an element of $A$.
          \end{proof}
          $z\mathfrak M = \frac{y}{x}\mathfrak M \subset \frac{\mathfrak M^r}{x} \subset A$.  We stop here (continue next time).
      \end{itemize}

    \end{proof}
\end {document}

