% Copyright 2006 by Till Tantau
%
% This file may be distributed and/or modified
%
% 1. under the LaTeX Project Public License and/or
% 2. under the GNU Public License.
%
% See the file doc/generic/pgf/licenses/LICENSE for more details.


% 
% This file provides utitiliy commands that are used throughout pgf.
%
% For most commands, the definition of these commands is just given
% below. We cannot use the LaTeX definition of these commands since
% LaTeX may not be the current format and since LaTeX packages tend to
% redefine these commands.
%
% For some commands the actual definition of the format (like latex or
% context) is to be preferred over the generic definition below. In
% this case, the definition of the format is installed when the file
% pgfutil-XXXX.tex is read, where XXXX is the format name (latex,
% plain, or context). 



\catcode`\@=11\relax


% Simple stuff

\long\def\pgfutil@ifundefined#1{%
  \expandafter\ifx\csname#1\endcsname\relax
    \expandafter\pgfutil@firstoftwo
  \else
    \expandafter\pgfutil@secondoftwo
  \fi}
\def\pgfutil@firstofone#1{#1}
\def\pgfutil@firstoftwo#1#2{#1}
\def\pgfutil@secondoftwo#1#2{#2}
\def\pgfutil@empty{}
\def\pgfutil@gobble@until@relax#1\relax{}
\def\pgfutil@gobble#1{}
\def\pgfutil@gobbletwo#1#2{}
\def\pgfutil@namedef#1{\expandafter\def\csname #1\endcsname}
\def\pgfutil@namelet#1{\expandafter\pgfutil@@namelet\csname#1\endcsname}
\def\pgfutil@@namelet#1#2{\expandafter\let\expandafter#1\csname#2\endcsname}
\long\def\pgfutil@g@addto@macro#1#2{%
  \begingroup
    \pgfutil@toks@\expandafter{#1#2}%
    \xdef#1{\the\pgfutil@toks@}%
 \endgroup}
\newif\ifpgfutil@tempswa

% pgfutil@ifnextchar

\long\def\pgfutil@ifnextchar#1#2#3{%
  \let\pgfutil@reserved@d=#1%
  \def\pgfutil@reserved@a{#2}%
  \def\pgfutil@reserved@b{#3}%
  \futurelet\pgfutil@let@token\pgfutil@ifnch}
\def\pgfutil@ifnch{%
  \ifx\pgfutil@let@token\pgfutil@sptoken
    \let\pgfutil@reserved@c\pgfutil@xifnch
  \else
    \ifx\pgfutil@let@token\pgfutil@reserved@d
      \let\pgfutil@reserved@c\pgfutil@reserved@a
    \else
      \let\pgfutil@reserved@c\pgfutil@reserved@b
    \fi
  \fi
  \pgfutil@reserved@c}
{%
  \def\:{\global\let\pgfutil@sptoken= } \:
  \def\:{\pgfutil@xifnch} \expandafter\gdef\: {\futurelet\pgfutil@let@token\pgfutil@ifnch}
}

% pgfutil@in@

\newif\ifpgfutil@in@

% Usage:
% \pgfutil@in@{one}{three two one}
% \ifpgfutil@in@
% 	-> will be true!
% \else
% \fi
%
% \pgfutil@in@{,}{1234,456567}
% \ifpgfutil@in@
% 	-> will be true!
% \else
% \fi
\def\pgfutil@in@#1#2{%
 \def\pgfutil@in@@##1#1##2##3\pgfutil@in@@{%
  \ifx\pgfutil@in@##2\pgfutil@in@false\else\pgfutil@in@true\fi}%
 \pgfutil@in@@#2#1\pgfutil@in@\pgfutil@in@@}


% pgfutil@for

\def\pgfutil@nnil{\pgfutil@nil}
\def\pgfutil@fornoop#1\@@#2#3{}
\long\def\pgfutil@for#1:=#2\do#3{%
  \expandafter\def\expandafter\pgfutil@fortmp\expandafter{#2}%
  \ifx\pgfutil@fortmp\pgfutil@empty \else
    \expandafter\pgfutil@forloop#2,\pgfutil@nil,\pgfutil@nil\@@#1{#3}\fi}
\long\def\pgfutil@forloop#1,#2,#3\@@#4#5{\def#4{#1}\ifx #4\pgfutil@nnil \else
       #5\def#4{#2}\ifx #4\pgfutil@nnil \else#5\pgfutil@iforloop #3\@@#4{#5}\fi\fi}
\long\def\pgfutil@iforloop#1,#2\@@#3#4{\def#3{#1}\ifx #3\pgfutil@nnil
       \expandafter\pgfutil@fornoop \else
      #4\relax\expandafter\pgfutil@iforloop\fi#2\@@#3{#4}}
\def\pgfutil@tfor#1:={\pgfutil@tf@r#1 }
\long\def\pgfutil@tf@r#1#2\do#3{\def\pgfutil@fortmp{#2}\ifx\pgfutil@fortmp\space\else
    \pgfutil@tforloop#2\pgfutil@nil\pgfutil@nil\@@#1{#3}\fi}
\long\def\pgfutil@tforloop#1#2\@@#3#4{\def#3{#1}\ifx #3\pgfutil@nnil
       \expandafter\pgfutil@fornoop \else
      #4\relax\expandafter\pgfutil@tforloop\fi#2\@@#3{#4}}


% pgfutil@IfFileExists

%\chardef\pgfutil@inputcheck0
\def\pgfutil@IfFileExists#1#2#3{%
  \openin\pgfutil@inputcheck=#1 %
  \ifeof\pgfutil@inputcheck
     #3\relax
  \else
    #2\relax
  \fi
  \closein\pgfutil@inputcheck}

\def\pgfutil@InputIfFileExists#1#2#3{\pgfutil@IfFileExists{#1}{\input #1\relax#2}{#3}}%


% pgfutil@loop (from plain.tex)

\def\pgfutil@loop#1\pgfutil@repeat{\def\pgfutil@body{#1}\pgfutil@iterate}
\def\pgfutil@iterate{\pgfutil@body \let\pgfutil@next\pgfutil@iterate \else\let\pgfutil@next\relax\fi \pgfutil@next}
\let\pgfutil@repeat=\fi % this makes \loop...\if...\repeat skippable


% aux-read-hook

\let\pgfutil@aux@read@hook=\relax


% Tokens for the end of the typesetting -- they will be added at the
% end of every job (hopefully...).

\newtoks\pgfutil@everybye


%
% PDF-Resource management -- might seem strange that this is here, but
% it turns out to be the correct place since latex and context handle
% this stuff quite differently
%

% Adding something to the pdf-resources:

\def\pgfutil@addpdfresource@extgs#1{\pgf@sys@addpdfresource@extgs@plain{#1}}
\def\pgfutil@addpdfresource@colorspaces#1{\pgf@sys@addpdfresource@colorspaces@plain{#1}}
\def\pgfutil@addpdfresource@patterns#1{\pgf@sys@addpdfresource@patterns@plain{#1}}
\def\pgfutil@setuppdfresources{\pgf@sys@setuppdfresources@plain}


% Inserts the argument at the begin of the current page (hopefully)

\let\pgfutil@insertatbegincurrentpage=\relax


% Library files inclusion


% Include a library file.
%
% #1 = List of names of library file.
%  
% Description:
%
% This command includes a list of library files. For each file X in the
% list, the file pgflibraryX.code.tex is included, provided this has
% not been done earlier. 
%
% For the convenience of Context users, both round and square brackets
% are possible for the argument.
%
% Example:
%
% \usepgflibrary{arrows}
% \usepgflibrary[patterns,snakes]

\def\usepgflibrary{\pgfutil@ifnextchar[{\use@pgflibrary}{\use@@pgflibrary}}%}
\def\use@pgflibrary[#1]{\use@@pgflibrary{#1}}
\def\use@@pgflibrary#1{%
  \edef\pgf@list{#1}%
  \pgfutil@for\pgf@temp:=\pgf@list\do{%
	\expandafter\pgfkeys@spdef\expandafter\pgf@temp\expandafter{\pgf@temp}%
	\ifx\pgf@temp\pgfutil@empty
	\else
		\expandafter\ifx\csname pgf@library@\pgf@temp @loaded\endcsname\relax%
		  \expandafter\let\csname pgf@library@\pgf@temp @loaded\endcsname=\pgfutil@empty%
		  \expandafter\edef\csname pgf@library@#1@atcode\endcsname{\the\catcode`\@}
		  \expandafter\edef\csname pgf@library@#1@barcode\endcsname{\the\catcode`\|}
		  \catcode`\@=11
		  \catcode`\|=12
		  \input pgflibrary\pgf@temp.code.tex
		  \catcode`\@=\csname pgf@library@#1@atcode\endcsname
		  \catcode`\|=\csname pgf@library@#1@barcode\endcsname
		\fi%
	\fi
  }%
}


% Include a module file.
%
% #1 = List of names of module files.
%  
% Description:
%
% This command includes a list of module files. For each file X in the
% list, the file pgfmoduleX.code.tex is included, provided this has
% not been done earlier. 
%
% For the convenience of Context users, both round and square brackets
% are possible for the argument.
%
% Example:
%
% \usepgfmodule{matrix}

\def\usepgfmodule{\pgfutil@ifnextchar[{\use@pgfmodule}{\use@@pgfmodule}}%}
\def\use@pgfmodule[#1]{\use@@pgfmodule{#1}}
\def\use@@pgfmodule#1{%
  \edef\pgf@list{#1}%
  \pgfutil@for\pgf@temp:=\pgf@list\do{%
    \expandafter\ifx\csname pgf@module@\pgf@temp @loaded\endcsname\relax%
      \expandafter\let\csname pgf@module@\pgf@temp @loaded\endcsname=\pgfutil@empty%
      \expandafter\edef\csname pgf@module@#1@atcode\endcsname{\the\catcode`\@}
      \expandafter\edef\csname pgf@module@#1@barcode\endcsname{\the\catcode`\|}
      \catcode`\@=11
      \catcode`\|=12
      \input pgfmodule\pgf@temp.code.tex
      \catcode`\@=\csname pgf@module@#1@atcode\endcsname
      \catcode`\|=\csname pgf@module@#1@barcode\endcsname
    \fi%
  }%
}

\def\pgfutilensuremath#1{%
  \ifmmode#1\else$#1$\fi
}


%
% Guess the driver:
%

\def\pgfutil@guessdriver{\edef\pgfsysdriver{pgfsys-\Gin@driver}}

\begingroup
	\catcode`\"=12
	\edef\pgf@loc@TMPa{"}%
	%
	% prepares file names by checking for double colons.
	%
	% If '#1' is a usual file name without anything fancy, \pgfretval
	% will simply contain it. However, if '#1' contains double colons
	% (introduces, for example, by pdftex because there are white
	% spaces in '#1'), the routine 
	% - removes the double colons,
	% - re-inserts them outside of the string.
	%
	% \pgfutilpreparefilename{file.tex} -> file.tex
	% \pgfutilpreparefilename{"A file name".file} -> "A file name.file"
	%
	% The resulting file name is returned in '\pgfretval',
	% furthermore, a *quoted* version of the file name is returned in
	% \pgfretvalquoted. The latter result is to have some sort of
	% output normalisation: if the file name as such contains double
	% quotes, we don't want to insert another set of them.
	%
	% This special handling has the following purposes:
	% 1. Both, treatment of white spaces and double colons is not
	% properly defined in TeX. Thus, we should only work with them if
	% necessary and maintain backwards compatibility as far as
	% possible.
	% 2. It *should* work if there are spaces.
	%
	\gdef\pgfutilpreparefilename#1{%
		\begingroup
			\ifnum\the\catcode`\"=13
				\pgfutilconvertdcolon
			\fi
			\xdef\pgf@temp{#1}%
		\endgroup
		\expandafter\pgfutil@in@\expandafter"\expandafter{\pgf@temp}%
		\ifpgfutil@in@
			\def\pgf@loc@TMPa{\pgfutilstrreplace{"}{}}%
			\expandafter\pgf@loc@TMPa\expandafter{\pgf@temp}%
			\edef\pgfretval{"\pgfretval"}% re-insert quotes! Otherwise, TeX can't use the file name.
			\let\pgfretvalquoted=\pgfretval
		\else
			\let\pgfretval=\pgf@temp
			\edef\pgfretvalquoted{"\pgfretval"}%
		\fi
	}%
	\catcode`\"=13
	\xdef\pgfutilconvertdcolon{%
		\noexpand\def\noexpand"{\pgf@loc@TMPa}%
	}%
\endgroup

% Usage:
% \pgfutilstrreplace{<token>}{<replacement>}{<string>}
%
% -> will assign the modified string into \pgfretval.
%
% #1: the string to search (one or more tokens)
% #2: zero, one or more tokens which will be inserted instead of '#1'.
% #3: the string to search in
\long\def\pgfutilstrreplace#1#2#3{%
	\def\pgfretval{}%
	\long\def\pgfutil@search@and@replace@@##1#1##2\pgf@EOI{%
		\expandafter\def\expandafter\pgfretval\expandafter{\pgfretval ##1#2}%
		\pgfutil@search@and@replace@loop{#1}{##2}%
	}%
	\pgfutil@search@and@replace@loop{#1}{#3}%
}
\long\def\pgfutil@search@and@replace@loop#1#2{%
	\pgfutil@in@{#1}{#2}%
	\ifpgfutil@in@
		\def\pgf@loc@TMPa{\pgfutil@search@and@replace@@ #2\pgf@EOI}%
	\else
		\expandafter\def\expandafter\pgfretval\expandafter{\pgfretval #2}%
		\let\pgf@loc@TMPa=\relax
	\fi
	\pgf@loc@TMPa
}%

% Solves a linear equation system of size 2x2 using gauss elimination.
%
% It employs TeX register arithmetics to do so.
% #1: should contain 4 sets of braces with matrix entries, 
% 	{<a11>}{<a12>}
% 	{<a21>}{<a22>}
% 	where each entry should be a number without unit.
% 	It is acceptable if '#1' is a macro which expands to the expected
% 	format.
% #2: should contain 2 sets of braces with the right-hand-side,
% 	{<r1>}{<r2>}
% 	where each entry should be a number without unit.
% 	It is acceptable if '#2' is a macro which expands to the expected
% 	format.
%
% It will assign \pgfmathresult to contain two sets of braces with the
% result.
%
% Example:
% \pgfutilsolvetwotwoleq{
% 	{0.24}{1}
% 	{-0.97}{0}
% }{
% 	{-7}
% 	{18}
% }
% -> yields \pgfmathresult={−18.55618}{−2.54642}
%
% The algorithm employs column pivotisation.
\def\pgfutilsolvetwotwoleq#1#2{%
	\begingroup
		\dimendef\aa=0
		\dimendef\ab=1
		\dimendef\ba=2
		\dimendef\bb=3
		\dimendef\ra=4
		\dimendef\rb=5
		\dimendef\tmpa=6
		\dimendef\tmpb=7
		\edef\pgf@temp{#1}%
		\expandafter\pgfutilsolvetwotwoleq@A\pgf@temp
		\edef\pgf@temp{#2}%
		\expandafter\pgfutilsolvetwotwoleq@r\pgf@temp
		%
		\pgfutilsolvetwotwoleq@ifislarger\aa\ba{%
			% identity "permutation":
			\def\Pa{a}%
			\def\Pb{b}%
		}{%
			% permutation matrix: switch rows!
			\def\Pa{b}%
			\def\Pb{a}%
		}%
		% \pivot := 1/aa
		\pgfmathreciprocal@
			{\csname m\Pa a\endcsname}%
		\let\pivot=\pgfmathresult
		%
		% \factor := 1/aa * ba 
		\csname \Pb a\endcsname=\pivot\csname \Pb a\endcsname
		\edef\factor{\expandafter\pgf@sys@tonumber\csname \Pb a\endcsname}%
		%
		% bb -= ba/aa * ab
		\tmpa=-\factor\csname \Pa b\endcsname
		\advance\csname \Pb b\endcsname by\tmpa
		%
		% rb -= ba/aa * ra
		\tmpa=-\factor\csname r\Pa\endcsname
		\advance\csname r\Pb\endcsname by\tmpa
		%
		% xb := rb / bb (the modified rb and modified bb!)
		\pgfmathdivide@
			{\expandafter\pgf@sys@tonumber\csname r\Pb\endcsname}
			{\expandafter\pgf@sys@tonumber\csname \Pb b\endcsname}%
		\expandafter\let\csname pgfmathresult\Pb\endcsname=\pgfmathresult
		%
		% ra -= xb * ab
		\tmpa=\csname pgfmathresult\Pb\endcsname\csname \Pa b\endcsname
		\advance\csname r\Pa\endcsname by-\tmpa
		%
		% xa := 1/aa * ra  (the modified ra!)
		\tmpa=\pivot\csname r\Pa\endcsname
		\expandafter\edef\csname pgfmathresult\Pa\endcsname{\pgf@sys@tonumber\tmpa}%
		%
		\edef\pgfmathresult{%
			{\csname pgfmathresult\Pa\endcsname}%
			{\csname pgfmathresult\Pb\endcsname}%
		}%
		\pgfmath@smuggleone\pgfmathresult
	\endgroup
}%
\def\pgfutilsolvetwotwoleq@ifislarger#1#2#3#4{%
	\tmpa=#1
	\ifdim\tmpa<0pt
		\multiply\tmpa by-1
	\fi
	\tmpb=#2
	\ifdim\tmpb<0pt
		\multiply\tmpb by-1
	\fi
	\ifdim\tmpa>\tmpb
		#3%
	\else
		#4%
	\fi
}%
\def\pgfutilsolvetwotwoleq@A#1#2#3#4{%
	\def\maa{#1}\def\mab{#2}%
	\def\mba{#3}\def\mbb{#3}%
	\aa=#1pt \ab=#2pt
	\ba=#3pt \bb=#4pt
}
\def\pgfutilsolvetwotwoleq@r#1#2{%
	\ra=#1pt \rb=#2pt
}%

% there are funny programs which overwrite \read and \write (like
% tabularx).
\let\pgfutil@write=\write
\let\pgfutil@read=\read

\endinput
